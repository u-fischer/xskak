% !Mode:: "TeX:DE:UTF-8:Main"
% -*- mode: TeX -*- -*- coding: UTF-8 -*-
\def\UFcurrentpackage{xskak}
\newcommand\UFcurrentversion{1.4}

\RequirePackage{fix-cm}

\documentclass[pagesize,parskip=half-,fontsize=12pt]{scrartcl}
\usepackage[utf8]{inputenc}

%% xskak specific commands
\usepackage[LSB1,LSB2, LSB3,LSBC2,LSBC3,LSBC4,LSBC5, T1]{fontenc}
\usepackage{xskak}
%\usepackage{eqparbox}
\usepackage{tabularx}

\usepackage{framed}
\colorlet{shadecolor}{lightgray!25}


\makeatletter
\newcommand\Describefillkey[4][]{\par\index{#2  (\textbackslash chessboard key)}%
\makebox[0pt][r]{\tiny #1\hspace{1em}}%
 \colorbox{green!25}{\makebox[\dimexpr\textwidth-2\fboxsep\relax]{\ttfamily%
    \eqparbox{fillkey}{#2=\meta{#3}}\hfill
    \eqparbox{fillexample}{%
     \ifthenelse{\equal{#4}{}}%
     {#2}
     {#2={#4}}}}}%
    \par\nobreak\@afterheading}
\makeatother

\usepackage{filecontents}
\usepackage{UF-xskak-documentation}

\setchessboard{smallboard, showmover=false}
\xskakset{style=UF}

\usepackage{showexpl}
%\input{I:/Z-test/showexpl-corr}
%\newcounter{showexpl}
%\renewcommand*\LTXinputExample[2][]{%
%  \stepcounter{showexpl}%
%  \g@addto@macro\SX@@explpreset{#1,codefile=#2}%
%  \SX@put@code@result}%
%
%\AtBeginDocument{%
%    \def\theHlstnumber{\ifx\lst@@caption\@empty \lst@neglisting
%                                          \else \thelstlisting \fi
%                       .\thelstnumber.\theshowexpl}}
\makeatletter
\renewcommand\board@pgf@extractangle[2]{%#1 command, #2 pgfpoint
   \pgfextracty{\len@board@tempy}{#2}%
   \pgfextractx{\len@board@tempx}{#2}%
  \ifdim\len@board@tempx=0pt\relax
  %\len@board@tempy= 1 or unchanged?
  \else
   %\divide \len@board@tempy by \strip@pt\len@board@tempx\relax
   \pgfmathsetlength{\len@board@tempy}{\len@board@tempy/\len@board@tempx}%
  \fi
   \ifdim\len@board@tempx=0pt\relax
     \ifdim\len@board@tempy=0pt\relax
      \def#1{0}%
     \else
      \ifdim\len@board@tempy>0pt\relax
       \def#1{90}%
      \else
       \def#1{-90}%
      \fi
     \fi
   \fi
   \ifdim\len@board@tempy=0pt\relax
    \ifdim\len@board@tempx=0pt\relax
      \def#1{0}%
    \else
      \ifdim\len@board@tempx>0pt\relax
       \def#1{0}%
      \else
       \def#1{180}%
      \fi
    \fi
   \fi
   \ifdim\len@board@tempx>0pt\relax
    \ifdim\len@board@tempy>0pt\relax
     \pgfmathatan{\len@board@tempy}%
     \edef#1{\pgfmathresult}%
    \fi
    \ifdim\len@board@tempy<0pt\relax
     \pgfmathatan{\len@board@tempy}%
     \edef#1{\pgfmathresult}%
    \fi
  \fi
  \ifdim\len@board@tempx<0pt\relax
    \ifdim\len@board@tempy>0pt\relax
     \pgfmathatan{-\len@board@tempy}%
     \pgfmathadd{-\pgfmathresult}{180}%
     \edef#1{\pgfmathresult}%
    \fi
    \ifdim\len@board@tempy<0pt\relax
     \pgfmathatan{\len@board@tempy}%
     \pgfmathadd{\pgfmathresult}{180}%
     \edef#1{\pgfmathresult}%
    \fi
  \fi
   }

\makeatother

\begin{document}

\title{\packagename{xskak}: An extension to \skaksty}
\author{Ulrike Fischer}
\maketitle

\tableofcontents

\section{Changes}
\begin{description}
\item[2015-01-02] (Version 1.4)] Corrected a bug: After a promotion the promotionpiece and the fen position were wrong.
\item[2014-04-19] (Version 1.3)] Corrected a bug: enpassant test didn't check the piece and so other figures than a pawn could make an enpassant move.
\item[2012-08-24 (Version 1.2b)]
Corrected a small bug -- \cs{printchessgame} doesn't wrap around a wrapfigure.


\item[2012-08-24 (Version 1.2a)]
Corrected a small bug -- \cs{hidemoves} printed comment signs.

\item[2008-10-10 (Version 1.2)]
Made some changes to make \xskak compatible to the version 1.5 of
\skaksty. See below for details.

Changed Code: As \cs{longmoves} doesn't work with \cs{variation}
anyway \cs{variation} now use always \cs{shortmoves}.

Bug correction: \skaksty forgets to store comments after castling
moves. Code to correct this has been added to \xskak.

Bug correction: \xskak didn't print check and mate symbol after
castling moves due to wrong position of braces.

\item[2008-07-28 (Version 1.1)]
Bug correction: Added the code for the promotion piece to the SAN and
LAN notation.

Bug correction: Remove some spurious spaces.

\end{description}

%\section{Features}
%
%

\section{Introduction}

\Xskak\footnote{License: \LaTeX\ Project Public License} is an extension for \skaksty -- and as \xskak is an extension
(and not a replacement) the main work is still done by \skaksty so
please read first the documentation of \skaksty as I'm not explaining
the \packagename{skak} commands in this documentation.


\Skaksty is a package from Torben Hoffmann which can parse and print
chess games given in (a subset of) the PGN notation.

The main features of \packagename{xskak} are:

\begin{itemize}
\item \packagename{xskak} can handle an extended set of PGN.

\item \packagename{xskak}
can parse and store an arbitrary number of chess games in PGN. The
games can be exported and imported and all positions and moves can be
retrieved at anytime.

\item \packagename{xskak} can
handle an arbitrary number of printing styles to distinguish
the different variations levels.


\end{itemize}





\minisec{Extended PGN handling}

When handling a PGN chess parsers at first have to split the notation
in different chunks. I will call this step the \emph{PGN parser}.
As an example\\[1ex]\mbox{}\quad {\ttfamily 15.Ne5!+
\$3 \{This breaks the black defense\} (15.Re1? Kg8!)}\\[1ex] contains as
chunks a \emph{move number} (\texttt{15.}), a \emph{move}
(\texttt{Ne5}), \emph{short move comments} (\texttt{!+}), a
\emph{NAG} (\texttt{\$3}), a \emph{comment} (\texttt{\{This breaks
the black defense\}}) and a \emph{variation} (\texttt{(15.Re1?
Kg8!)}).

Neither \packagename{skak} nor \packagename{xskak} can handle all
legal PGN's. \Skaksty can only handle PGN's which contains move
numbers, moves and short comments. \packagename{xskak} can
additionally handle NAG's and a special form of comments. None of the
packages can currently parse PGN's with variations.


\minisec{Storing moves}

After the splitting of the PGN the chunks are processed further.
Depending on the type of the chunk different actions are taken. The
most important subprocess is the \emph{move parser}. It takes the
move, retrieves from somewhere the current (FEN-)position of the game
and tries to figure out what the move is exactly doing.

In \packagename{xskak} only the commands \cs{mainline} and
\cs{hidemoves} calls the move parser, \cs{variation} uses only the
PGN parser.




The main difference between \packagename{xskak} and
\packagename{skak} is how they handle the information gathered by
the move parser: \Skaksty uses the results of the move parser mainly
to update the board and after the move throws almost all other
information away -- including previous positions. \xskak saves all
information about the moves for later use. Here a simple example:

\begin{LTXexample}
 \newchessgame
 \mainline{1. e4 e5 2. Nf3 Nc6 3. Bb5 a6 4. Ba4 Nf6}

 \xskakset{moveid=2w}%
 \chessboard[setfen=\xskakget{nextfen}]\\[1ex]
 Position after 2.\,\xskakget{lan}
\end{LTXexample}

So in short you start a game with \cs{newchessgame}, you enter moves
with \cs{mainline} (or \cs{hidemoves}), you set the move for which
you want to retrieve data with \cs{xskakset}, and then you retrieve
the data with \cs{xskakget}.


The storing of the moves and positions is not restricted to the main
game. You can store and retrieve as much variation levels you want.


\minisec{Printing styles}

The following example shows how to use different styles for different
variation levels. But it also shows the problems: It can be quite
tiresome to get a good printing result. You must pay attention to the
levels and the spaces after commands, and for each variation you have
to start a new game and resume older games if you go back to them.

\begin{LTXexample}
\longmoves \xskakset{style=UF}
\newchessgame[id=main]
\mainline{1.e4 e5}
%
 (Alternatives are
 \newchessgame[newvar=main,id=var1]%
 \mainline[invar]{1... d5} which is answered by
 \mainline{2. exd5 Qxd5}
  %
   \newchessgame[newvar=var1,id=var1-1]
   (\mainline[invar]{2... Nf6})
   %
 or
 \newchessgame[newvar=main,id=var2]
 \mainline[outvar]{1... e6 2. d4})
%
\resumechessgame[id=main]%
\mainline[outvar]{2.Nf3}

\chessboard
\end{LTXexample}



\bigskip

In the following sections I will describe the finer points. This
includes
\begin{itemize}
\item how to handle more than one game at a time,
\item how to retrieve the data faster,
\item what data you can retrieve,
\item how to loop through a game,
\item how to export the data,
\item how to improve the printing of the games,
\item and some other changes/additions I made to \packagename{skak}.
\end{itemize}


\subsection{Warnings}

\minisec{\packagename{xskak} and \packagename{skak}}

\Xskak doesn't only define some new commands, it also changes some
quite central internal commands of \skaksty. This means
\begin{itemize}
    \item it can break standard commands of \skaksty,

    \item it will possibly not work together with future versions of
    \skaksty,

    \item loading \packagename{xskak.sty} can lead to problems with
    packages like \packagename{texmate} which use internally
    \packagename{skak}.
\end{itemize}

\minisec{Local and global commands}

In the package \packagename{chessboard} almost every setting is local
to the board. This was a quite natural decision as a board is a
confined space. In \packagename{xskak} the situation is quite
different: As a start \packagename{skak} already sets and saves some
information globally and the data saved by \packagename{xskak}
during the parsing must be saved globally or they wouldn't never
escape the \cs{mainline}. And secondly, as I wanted to use
\packagename{beamer} and \packagename{animate} a lot of data
must be preserved from one frame/picture to the next. So I decided to
use in most cases global commands unless -- like in the cases of
styles -- I could see a real advantage in keeping a setting local.

So please be aware that settings made e.g. in a figure environment
can affect the following text.


\subsection{Some history}

While the commands \cs{mainline} and \cs{hidemoves} parse a chess
game \skaksty updates an internal board and so can always show the
current position. Sadly a lot of the data gathered during the
parsing are thrown away after \packagename{skak} has updated the
board and  has go on to the next move and so can not be used by the
user. E.g. there is no way to get the field a piece came from or to
store easily all positions of a game.

Some years ago I tried to find out how to print a game in long
algebraic notation and I realized quite fast that I need some of
these lost data to be able to do it, and that it would be necessary
to patch some internal commands of \skaksty to save the data. At the
same time I also tried to insert some code that allows the parsing of
standard PGN-comment signs. Due to lack of time the whole project
didn't got very far but I had a working beta. Then a new version of
\skaksty came. It had even a -- sadly not fully working (as can be
seen in example~\ref{ex:longmoves}) -- command \cs{longmoves}%
    \index{longmoves=\verb!*+\longmoves+>}%
    \index{shortmoves=\verb!*+\shortmoves+>}
(and the counterpart \cs{shortmoves}) to switch to the long algebraic
notation. But I hadn't the time to test and adjust my package.

    \index{longmoves=\verb!*+\longmoves+>}%
    \index{shortmoves=\verb!*+\shortmoves+>}%
\LTXinputExample[label=ex:longmoves,
                   graphic={[trim=3cm 24cm 9cm
                   4cm,clip]{skak-longmoves-test}},
                   caption={Wrong pawn moves in the
                            longmoves implementation of \skaksty version 1.4}]{skak-longmoves-test}




Then in 2006 I wrote \pchessboard and during the tests also tried to
make animated pdf's which show all positions of a game. That worked
fine but the input was so tiresome that I restarted the
\packagename{xskak}-project: Like in the case of the long algebraic
notation the main problem was the loss of previous positions.


\subsection{Bugs and errors}

I'm quite sure that they are bugs and errors in the package.

If you have questions ask them in the newsgroups
\nolinkurl{comp.text.tex}, \nolinkurl{de.comp.text.tex} or
\nolinkurl{fr.comp.text.tex} or at \url{http://tex.stackexchange.com/}. I'm reading these groups regularly and
I'm much better in answering there than in answering e-mails.

If you find errors in this text (this includes wrong english) or in
the package, you can write me a bugreport at
\nolinkurl{chess@nililand.de}. A bugreport should contain a complete,
running, \emph{minimal} example and the log-file of the pdf\LaTeX\
run (that means the engine that makes a pdf!).


\subsection{Requirements}

\Xskak uses some primitives of e\TeX. It needs version 1.4 or version
1.5 of \skaksty, a recent version of \chessfss (chess font selection)
and \packagename{xkeyval} (\textit{key=value}-syntax), version 1.5 of
\packagename{chessboard} (which in turns needs \packagename{pgf}) and
\packagename{xifthen}.


\subsection{Installation}

Run \filetype{xskak.ins} through \LaTeX\ and then put the
\filetype{*.sty} where \LaTeX\ finds them.
(\path{<texmf>/tex/latex/xskak/} is a good place.) Update the
filename database.


\subsection{Loading and package options}

Loading of the package is done with \verb+\usepackage{xskak}+.
Currently the package has only one option: \texttt{nopdfmatch}.%
    \index{nopdfmatch (package option)}
The option will disable the use of the experimental \cs{pdfmatch}
used in the  processing of the style levels (see section
\ref{sec:usestyles}). \packagename{xskak} will load \skaksty without
option (and won't pass options to it). So, if you want to load
\packagename{skak} with options, you must do it before loading
\xskak.

\subsection{Adaption to \skaksty version 1.5}

Above I warned that \xskak can break with newer versions of \skaksty.
Some days ago the version 1.5 of \packagename{skak} appeared and
promptly a lot of things broke so I had to adapt and change my code.
Version 1.2 of \xskak will hopefully work with both versions of
\skaksty.

In version 1.5 of \skaksty some fundamental things have changed. At
first \cs{longmoves} works now correctly, even as you can see for the
\cs{variation} command:


\LTXinputExample[label=ex:longmovesnew,
                   graphic={[trim=3cm 24cm 9cm
                   4cm,clip]{skak-longmoves-test-new}},
                   caption={The new
                            longmoves implementation of
                            \skaksty}]{skak-longmoves-test-new}


The more important change (from the point of my package) is that
\cs{variation} now parse the moves too (that's why it can show the
long algebraic notation). To achieve this the author of \skaksty,
Torben Hoffman, stores now the position before a move so that he can
go back to this position and start the variation from there.

The main problem with this approach is that it imposes restrictions
on the move numbers of variations: All variations must start with the
move number of the last move in a \cs{mainline} command. So
separating two moves of a variation by some text is no longer
possible:

\begin{lstlisting}
 \documentclass{article}
 \usepackage{skak15}  %=skak version 1.5
 \begin{document}
 \newgame
 \mainline{1.e4 e5 2. Nf3}
 \variation{2. d3 d6} %works
 \variation{2. c3}    %works
 and then
 \variation{2...c6}   %error
 \end{document}
\end{lstlisting}

To tell the truth, I'm not sure that this a very good idea. In any
case: \xskak overwrites the responsable definitions of \skaksty. With
\xskak \cs{variation} currently doesn't parse games (and so also
can't show moves in long algebraic notation -- if you want this you
will have to start with the commands described later in this
documentation a new game with another id). So the above example works
fine with \xskak.





\subsection{The example}

Almost all examples use as starting point  a new game. But in some
cases the following position is used, which allows to show almost all
special cases (like enpassant, promotion, capture, castling,
ambiguous moves). The position is stored under the name «example» and
so can be retrieved e.g. with the key \key{restorefen=example} from
\packagename{chessboard} or the command
\lstinline+\restoregame{example}+ of the package
\packagename{skak}\footnote{One could also use  key \key{tag} to
store the position and key \key{reffen} to retrieve it.}:

\begin{LTXexample}[caption={The standard example position},width=.7\textwidth]
 \newchessgame[id=example,
   castling=Q,
   setwhite={pa2,pc7,pg5,ph2,pe4,
             ke1,ra1,bg4},
   addblack={pb4,ph7,pd5,
             ke8,rh8,bh3,nf6,ne5},
   storefen=example]%

 \mbox{\Xskakexampleinitfen}

 \chessboard
\end{LTXexample}


\section{Improved parsing of comments}\label{sec:diagrams}

 \Skaksty can handle comments in the input if they follow
directly the move, if the commands are robust and if the spaces that
separate moves and numbers are not swallowed by a command.

\LTXinputExample[label=ex:comments,
                   graphic={[trim=3cm 24cm 9cm
                   4cm,clip]{skak-comment-test}},
                   caption={comments in \skaksty}]{skak-comment-test}


This is quite okay for a few small comments in a small number of
games, but a bit tiresome if you want to handle a lot of games that
you have annotated with a chess application.  The PGN export of such
games uses NAG's for short comments and curly braces for longer
comments.


With \xskak the commands \cs{mainline}, \cs{hidemoves} and
\cs{variation} now can parse NAG's and longer comments if the input
meet the following conditions:
%\end{document}
\begin{itemize}
     \item There must be  spaces between the moves, each NAG and the
     comments.

     \item Text comments must be written as
     \verb+\xskakcomment{+\meta{Text}\verb+}+%
      \footnote{Braces are so frequent in \TeX\ that is simply not
      possible to identify the comments simply by looking for a brace
      -- at least not without a lot of complicated hacks.}

\end{itemize}

While parsing the comments the spaces before and between them will be
swallowed.  If you want a space you must add it to the text. The
NAG's (\$1, \$2 etc) are translated  following the definitions in
\filetype{xskak-nagdef.sty}. The list there is neither complete nor
definite. For quite a lot NAG's I hadn't the faintest idea what do.
I'm open for suggestions/corrections.

There exist two «fake» NAG's: \$d and \$D. If used in a \cs{mainline}
or \cs{hidemoves} both will save the identification of the current
move in an internal list and print whatever they have been defined
to.  (Currently \$D will print \cs{chessdiagramname} which is defined
as \verb+\ (Diagram)\ +, \$d is
empty). This can be used to refer to diagrams printed later.%
  \footnote{It is not a accidental that there is a similar command
  «\cs{toD}» in \packagename{texmate}. That's where I got the idea
  from.}



\begin{LTXexample}[caption={NAG's, comments and diagram references in the input}]
 \newchessgame[id=GameA]
 \mainline{1.e4! $2 e5  $D 2.Nf3 Nf6 \xskakcomment{ What's that? }}

 \renewcommand\xskakcomment[1]{{\itshape #1}}%
 \mainline{3. Nxe5
   \xskakcomment{ I took a pawn! } $D }
 \variation[invar]{3. Nc3 $4
   \xskakcomment{ (boring) }}

 \makeatletter
 \@for\mymoveid:=\XskakGameAdiagramlist\do{%
  \xskakset{moveid=\mymoveid}%
  \begin{tabular}{c}
  \chessboard[tinyboard,
      setfen=\xskakget{nextfen}]\\
  After move \xskakget{opennr}\xskakget{lan}
  \end{tabular} }
\end{LTXexample}


\section{The core of \xskak: Saving games.}

\Xskak redefines some internal commands of \skaksty to enable
\cs{mainline} and \cs{hidemoves} to store quite a lot%
 \footnote{\textbf{Attention:} When I say «a lot» I really mean it. Currently
  about 30 commands for each halfmove is saved plus about 30 commands
  for each game.  A lot of games with individual game identifications
  can use a lot of memory. But in practice that shouldn't be a problem
  as it is seldom necessary to use a lot of different game
 identifications.}
data about the moves and the state of the game while parsing
the game.

There are two sorts of data:  Information related to the game as a
whole (e.g. the names of the players and the start position) and
information related to the moves. Both type of information use a
\meta{GameId} in the names of the commands to identify the game they
are referring to, the move data use additionally a \meta{MoveId}.

\begin{shaded}
\minisec{Exkurs 1}

It is important to understand one difference between the use of
\meta{GameId} and \meta{MoveId}: There is only \emph{one} command
that holds the current \meta{GameId}. The \meta{GameId} is a concept
specific to \xskak. The current value \meta{GameId} is stored in a
command and will be used (and can be changed) by all the commands
that store information \emph{and} by all the commands that retrieve
information. So if you are parsing a game in bits you should be
careful to use the same \meta{GameId} for each part.

On the other sides there are \emph{three} different places where move
data are set or used: \skaksty uses in \cs{mainline} and
\cs{hidemoves} a counter called «\texttt{move}» and a boolean to keep
track of the current move and player. \Xskak has two internal
commands which are used to set up the start moves of games and to
retrieve the stored move data, and \cs{chessboard} has two keys that
set move data in FEN's. This can be sometimes a bit
confusing. \Skaksty will not like it if its move counter is changed
while parsing a game, but it is  normally not a problem to change the
\packagename{xskak} move commands e.g. to retrieve data about
previous moves -- as long as you don't use commands like
\cs{newchessgame} that affects also the \packagename{skak} move
counter.
\end{shaded}


\begin{shaded}
\minisec{Exkurs 2}

In chess we speak quite as often about the last move and the last
position than about the next move and the next position: A position
can be described as the one \emph{after} the \emph{last} move, the
result of a move. But it can also be described as the one
\emph{before} the \emph{next} move.

The keys and commands described later reflect this: Given a move
number you can access e.g. the position \emph{before} the move
through the key word \key{pastfen} and the position \emph{after} the
move through the key word \key{nextfen}. Given a \emph{tag} -- that
is a label between two moves, you can refer to the move
\textit{before} the tag through keys with the prefix \key{past} and
to the move \emph{after} the tag through keys with the prefix
\key{next}.

In short: \key{past} looks back and \key{next} looks forward.

The first position of a game and the first move have the prefix
\key{init}. The last position (after the last move) and the last move
have the prefix \key{last}.

\end{shaded}


\subsection{Initialization of the game}

\DescribeMacro{\newchessgame}%
When using \xskak you should start a new game with
\cs{newchessgame}\keyoarg{} instead of \cs{newgame}. It will set the
identification \meta{GameId} of the game, the start position, start
move and quite a lot of other variables to sensible values.

In the optional argument you can use the following keys:

\begin{description}
\item[\key{id=}\meta{GameId}]
    \index{newchessgame=\verb!*+\newchessgame+>id (key)}%
    \index{id (\texttt{\textbackslash newchessgame} key)}%
    This key sets the string that identifies the game and is used in
    the following parsing commands to store the data.
    It is possible to use commands and counters e.g. for automatic numbering of
    the games.
    If you don't use the key or if you use the key without
    argument (\verb+\newchessgame[id]+) a default is used.
    The initial value of this default is \texttt{game}, but it can be changed with
    \cs{xskakset}.
    You can use the key \key{id} in each \cs{newchessgame} only once.

\begin{LTXexample}[varwidth=false,width=0.3\textwidth]
 \newchessgame[white=Bob]%
 \mainline{1. e4 e5}

 \newchessgame[id=MY,white=Bill]%
 \mainline{1. e4 e5}

 \newcounter{gamecounter}%
 \stepcounter{gamecounter}%
 \xskakset{defaultid=Game\Alph{gamecounter}}%

 \newchessgame[white=Harry]%
 \mainline{1. e4 e5}

 \stepcounter{gamecounter}%
 \newchessgame[white=Thomas]%
 \mainline{1. d4 d5}

 \Xskakgamewhite,  \XskakMYwhite,
 \XskakGameAwhite, \XskakGameBwhite%
\end{LTXexample}

\item[\key{movenr=}\meta{number}, \key{player=}\meta{«w» or «b»}, \key{moveid=}\meta{number + «w» or «b»}]
    \index{newchessgame=\verb!*+\newchessgame+>movenr (key)}%
    \index{movenr (\texttt{\textbackslash newchessgame} key)}%
    \index{newchessgame=\verb!*+\newchessgame+>player (key)}%
    \index{player (\texttt{\textbackslash newchessgame} key)}%
    \index{newchessgame=\verb!*+\newchessgame+>moveid (key)}%
    \index{moveid (\texttt{\textbackslash newchessgame} key)}%
     With this keys you can change the start move for the parsing
     commands \cs{mainline} and \cs{hidemoves}.
     Setting this keys will change the counter and boolean
     of \packagename{skak} \emph{and} the internal commands of
     \packagename{xskak}.
     The changes are global.
     Default is \texttt{moveid=1w}, this default can be changed with
     \cs{xskakset}.
     Move number and player given through one of the FEN-keys of \cs{chessboard}
     will not affect the move counter of \packagename{skak}!


\begin{LTXexample}[varwidth=false,width=0.3\textwidth]
 \newchessgame
 \mainline{1. e4}

 \newchessgame[moveid=4w]
 \mainline{4. e4 e5}

 \xskakset{defaultmoveid=5b}
 \newchessgame
 \mainline{5... c4}
 \xskakset{defaultmoveid=1w} %back to normal
\end{LTXexample}

\item[\key{newvar=}\meta{GameId}]
    \index{newchessgame=\verb!*+\newchessgame+>newvar (key)}%
    \index{newvar (\texttt{\textbackslash newchessgame} key)}%

     If you use this key \cs{newchessgame} will use the position
     and the \meta{moveid} \emph{before} the last
     move of the game \meta{GameId} as start position. If you don't
     give a value the current meaning of \meta{GameId} is used. This
     can be used to start variations. The new game inherits only the
     start position of the reference game!


\item[\key{ref\ldots=}\meta{tag}]

The \meta{GameId} and the start move can also be set with keys which
use values saved with a \key{tag}. The keys are described in section
\ref{sec:labeling}.


\item[keys to set PGN-infos]
    \index{newchessgame=\verb!*+\newchessgame+>result (key)}%
    \index{result (\texttt{\textbackslash newchessgame} key)}%
    \index{newchessgame=\verb!*+\newchessgame+>white (key)}%
    \index{white (\texttt{\textbackslash newchessgame} key)}%
    \index{newchessgame=\verb!*+\newchessgame+>black (key)}%
    \index{black (\texttt{\textbackslash newchessgame} key)}%
    \index{newchessgame=\verb!*+\newchessgame+>whiteelo (key)}%
    \index{whiteelo (\texttt{\textbackslash newchessgame} key)}%
    \index{newchessgame=\verb!*+\newchessgame+>blackelo (key)}%
    \index{blackelo (\texttt{\textbackslash newchessgame} key)}%
    \index{newchessgame=\verb!*+\newchessgame+>site (key)}%
    \index{site (\texttt{\textbackslash newchessgame} key)}%
    \index{newchessgame=\verb!*+\newchessgame+>date (key)}%
    \index{date (\texttt{\textbackslash newchessgame} key)}%
    \index{newchessgame=\verb!*+\newchessgame+>event (key)}%
    \index{event (\texttt{\textbackslash newchessgame} key)}%
    \index{newchessgame=\verb!*+\newchessgame+>round (key)}%
    \index{round (\texttt{\textbackslash newchessgame} key)}%
    \index{PGN-infos |see (\texttt{\textbackslash newchessgame} keys)}%
    With the keys \key{result}, \key{white}, \key{black},
    \key{whiteelo}, \key{blackelo}, \key{site}, \key{date},
    \key{event} and \key{round} you can store the content of the
    PGN-infos. Unused keys are set to default values. These defaults
    can be changed with \cs{xskakset}.

    \DescribeMacro{\xskaknewpgninfo}%
    With \cs{xskaknewpgninfo}\oarg{default value}\marg{keyname} you can define a own PGN-info key. This
    will also define a key \key{default\meta{name}} to change the default value later.

\begin{LTXexample}[varwidth=false,width=0.3\textwidth]
 \xskaknewpgninfo[Tom]{commentary}

 \newchessgame[white=Harry,date=2007]
 \xskakgetgame{white}, \xskakgetgame{date},
 \xskakgetgame{event},
 \xskakgetgame{commentary}

 \xskakset{defaultevent=My open}

 \newchessgame[commentary=Müller]
 \xskakgetgame{commentary},
 \xskakgetgame{event}

\end{LTXexample}


\item[using chessboard keys]
    \index{newchessgame=\verb!*+\newchessgame+>using chessboard keys}%
    You can use any key that can be used in the optional argument of
    \cs{chessboard} from the package \packagename{chessboard}.
    \cs{newchessgame} will pass all keys to an internal
    \cs{chessboard}%
     \footnote{That's why you will get an error from \cs{chessboard} if you use an unknown key.}.
    \cs{chessboard} will set a position and give it back as the start position for the
    game.
    All \packagename{chessboard} keys are processed normally.
    You can even print directly the board by using the key \key{print}.
    The only exceptions\footnote{They don't gives
    error but they will have no effect.} are the keys \key{maxfield}
    and \key{zero} that change the size of the board --
    \packagename{skak} doesn't like unusual sizes of the board --
    and the keys \key{mover} and \key{fullmove} which are overwritten
    by the (eventually default) value of key \key{moveid}. The castling
    possibilities and the enpassant field are set accordingly the
    values in the default fen, you only need to set them if you want
    to use other values.
    Read the documentation of \packagename{chessboard} for more information.


\begin{LTXexample}
\newchessgame[
  moveid=4b,print,
  showmover,
  mover=w,% has no effect
  castling=Q,enpassant=a3,
  setwhite={pa4,pc7,pg5,ph2,pe4,ke1,ra1,bg4},
  addblack={pb4,ph7,pd5,ke8,rh8,bh3,nf6,ne5}]%

\mainline{4...  bxa3}
\chessboard
\end{LTXexample}
\end{description}

    To use a position of another game as a start position you can use
    e.g. this sequence of keys
    (a shorter method that use a label/ref-method will
    be described in the section \ref{sec:labeling}):

    \cs{newchessgame}\verb+[id=+\meta{old id}%
    \verb+, moveid=+\meta{old move id}%
    \verb+,+\\\phantom{(newchessgame[}%
    \verb+setfen=\xskakget{+\meta{\ttfamily\upshape nextfen|pastfen}\verb+}+%
    \verb+,+\\\phantom{(newchessgame[}%
    \verb+id=+\meta{new id}\verb+, moveid=+\meta{new initmove id}%
    \verb+]+

    \textbf{Attention:} A new game is a \emph{new} game. It doesn't inherit moves of
    another game -- it only reuses a position.
    So if the new and the old game have identical \meta{GameId}
    then the old game will be more or less lost.

   In case that you want to go one half move back and start a new
   game from there you can use the key \key{newvar}:

   \cs{newchessgame}\verb+[newvar=+\meta{old id}%
                    \verb+, id=+\meta{new id}\verb+]+


\begin{LTXexample}
 \newchessgame[id=A]%
 \mainline{1. e4 e5 2. Nf3 Nc6 3. Bb5 a6 4. Ba4 Nf6 5. O-O}

 \newchessgame[
     id=A,moveid=3w,print,
     setfen=\xskakget{nextfen},
        %fen after move 3w
     id=B,moveid=1b]

 \mainline{1... Nf6}

 \newchessgame[id=C]
 \mainline{1. e4 e5 2. Nf3 Nc6 3. Bb5 a6 4. Ba4 Nf6 5. O-O}

 \newchessgame[print,
       id=C,
  % we retrieve the number of the last move
  % of game C and set it as first number
  % of the new game B at the same time:
       moveid=\xskakgetgame{lastmoveid},
       setfen=\xskakget{pastfen},
       id=B]

 \mainline{5. d3}
\end{LTXexample}


\subsection{Continuing a game}

\DescribeMacro{\resumechessgame}%
    \Skaksty doesn't have a problem if you split a game in pieces.
    You can use as many \cs{mainline} commands as you want. It even
    doesn't matter if you started the game in another group. But if
    you mix two games you must first reset the
    position and the move counter to the end position of the old game
    before you can continue it. This is possible with
    \cs{resumechessgame}\keyoarg. You can use
    \cs{chessboard}-keys in the optional argument but all the keys
    that tries to change the position will have no effect. It is also
    not possible to store the new moves under a new game identification or
    to renumber the moves.

    The command accepts the following additional keys in the optional
    argument:

\begin{description}
\item[\key{id=}\meta{GameId}]
    \index{resumechessgame=\verb!*+\resumechessgame+>id (key)}%
    \index{id (\texttt{\textbackslash resumechessgame} key)}%
    This key identifies the game that should be continued (as a
    default the currently active identification is used). The key
    changes \meta{GameId} globally!


\item[\key{movenr=}\meta{number}, \key{player=}\meta{«w» or «b»}, \key{moveid}\meta{number + «w» or «b»}]
    \index{resumechessgame=\verb!*+\resumechessgame+>movenr (key)}%
    \index{movenr (\texttt{\textbackslash resumechessgame} key)}%
    \index{resumechessgame=\verb!*+\resumechessgame+>player (key)}%
    \index{player (\texttt{\textbackslash resumechessgame} key)}%
    \index{resumechessgame=\verb!*+\resumechessgame+>moveid (key)}%
    \index{moveid (\texttt{\textbackslash resumechessgame} key)}%
    This keys set the move from which on the game should be
    restarted. With the keys it is possible to go back and then
    continue the game along another variation (in this case the moves
    of the first variation will be lost).
    The values describes the \emph{next} move that means the position \emph{before} the move is
    used.
    As a default \cs{resumechessgame} continues the game.

\begin{LTXexample}[width=0.5\textwidth,preset=\parskip1ex]
 \newchessgame[id=main]
 \mainline{1. e4 e5 2. Nf3 Nc6 3. Bb5}

 That reminds me another game
 some years ago where I played
 \newchessgame[id=exkurs]
 \mainline{1. d4 Nf6 2. c4 e6}
 \ldots\ lots of text \ldots\

 Let's continue the second game with --
 as a reminder -- the last move of
 black
 \resumechessgame[id=exkurs,moveid=2b]%
 \mainline{2... e6 3. Nc3}

 Let us now go back to the first game\\
 \resumechessgame[id=main,print,
                 tinyboard,showmover]%
 \mainline{3... a6}
\end{LTXexample}

\item[\key{newvar=}\meta{GameId}]
    \index{resumechessgame=\verb!*+\resumechessgame+>newvar (key)}%
    \index{newvar (\texttt{\textbackslash resumechessgame} key)}%

     This key moves you one halfmove back.
     This can be used to write down a lot of variations. With the
     (optional) value of the key you set the \meta{GameId}. If you don't
     give a value the current meaning of \meta{GameId} is used.

\begin{LTXexample}[width=0.5\textwidth,preset=\parskip1ex]
 \newchessgame[id=main]
 \mainline{1. e4 e5 2. Nf3 Nc6 3. Bb5}


 \newchessgame[id=exkurs]
 \mainline{1. d4 Nf6 2. c4 e6}

 \resumechessgame[newvar=main]
 \mainline{3. Bc4}


 \resumechessgame[newvar]
 \mainline{3. Nc3}
 \chessboard

 \xskakset{id=exkurs}
 \resumechessgame[newvar]
 \mainline{2... c5}
 \chessboard

\end{LTXexample}

\item[\key{ref\ldots=}\meta{tag}]
The \meta{GameId} and start move can also be set with this keys which
use values saved with a \key{tag}. The keys are described in section
\ref{sec:labeling}.

\end{description}

\subsection{The stored information}

The following lists describes all the types of information stored
during the parsing of a game. As you will see some data are stored
more than once with only slight differences. I did it because it is
easier to store the redundant data than to have to convert them
later.

There are also commands which contains the \meta{GameId} and the
\meta{MoveId}. That looks a bit senseless as you need to know this
values to be able to retrieve them but they allow easy access to the
values without having to know or use the internal command names.


\subsubsection{Essential game related information}

\begin{description}

\item[\textbackslash Xskak\meta{GameId}initfen]%
    \index{game data >Xskakinitfen=\textbackslash Xskak\meta{GameId}initfen}%
    \index{Xskakinitfen=\textbackslash Xskak\meta{GameId}initfen}%
    The position before the first move. Set by \cs{newchessgame}.


\item[\textbackslash Xskak\meta{GameId}initplayer]%
    \index{game data >Xskakinitplayer=\textbackslash Xskak\meta{GameId}initplayer}%
    \index{Xskakinitplayer=\textbackslash Xskak\meta{GameId}initplayer}%
     «w» or «b». The color of the next (first) player, (must be identical
     to the color in the initfen). Also set by \cs{newchessgame}.

\item[\textbackslash Xskak\meta{GameId}initmovenr]%
    \index{game data >Xskakinitmovenr=\textbackslash Xskak\meta{GameId}initmovenr}%
    \index{Xskakinitmovenr=\textbackslash Xskak\meta{GameId}initmovenr}%
     The number of the first move (must be identical to the fullmove
     number in the initfen). Also set by \cs{newchessgame}.

\item[\textbackslash Xskak\meta{GameId}initmoveid]%
   \index{game data >Xskakinitmoveid=\textbackslash Xskak\meta{GameId}initmoveid}%
   \index{Xskakinitmoveid=\textbackslash Xskak\meta{GameId}initmoveid}%
   Combination of \key{initmovenr} and \key{initplayer}.

\begin{LTXexample}
 \newchessgame[id=show,
               setfen=2K/k/7R,
               moveid=10b,
               print]

 \mbox{\xskakgetgame{initfen}},\\
 \xskakgetgame{initplayer},
 \xskakgetgame{initmovenr},
 \xskakgetgame{initmoveid}
\end{LTXexample}



\item[\textbackslash Xskak\meta{GameId}lastfen]%
   \index{game data >Xskaklastfen=\textbackslash Xskak\meta{GameId}lastfen}%
   \index{Xskaklastfen=\textbackslash Xskak\meta{GameId}lastfen}%
   The fen \emph{after} the last move, set during parsing.

\item[\textbackslash Xskak\meta{GameId}lastplayer]%
   \index{game data >Xskaklastplayer=\textbackslash Xskak\meta{GameId}lastplayer}%
   \index{Xskaklastplayer=\textbackslash Xskak\meta{GameId}lastplayer}%
    «w» or «b». The player which made the last move, set by
    \cs{newchessgame}, by \cs{resumechessgame} and during
    parsing. Identical to color in the fen \emph{before} the last
    move! \emph{Not} identical to the color in {\texttt\textbackslash
    Xskak\meta{GameId}lastfen}!

\item[\textbackslash Xskak\meta{GameId}lastmovenr]%
   \index{game data >Xskaklastmovenr=\textbackslash Xskak\meta{GameId}lastmovenr}%
   \index{Xskaklastmovenr=\textbackslash Xskak\meta{GameId}lastmovenr}%
   The number of the last move, set during parsing. Identical to
   fullmove in the fen \emph{before} the last move! \emph{Not}
   necessarily identical to the number in {\texttt\textbackslash
   Xskak\meta{GameId}lastfen}!

\item[\textbackslash Xskak\meta{GameId}lastmoveid]%
   \index{game data >Xskaklastmoveid=\textbackslash Xskak\meta{GameId}lastmoveid}%
   \index{Xskaklastmoveid=\textbackslash Xskak\meta{GameId}lastmoveid}%
   Combination of \key{lastmovenr} and \key{lastplayer}.

\begin{LTXexample}
 \newchessgame[id=show,
               setfen=2K/k/7R,
               castling={},
               % sets correct castling in the FEN
               moveid=10b]
 \mainline{10... Ka8}

 \chessboard

 \mbox{\xskakgetgame{lastfen}},\\
 \xskakgetgame{lastplayer},
 \xskakgetgame{lastmovenr},
 \xskakgetgame{lastmoveid}
\end{LTXexample}


\item[\textbackslash Xskak\meta{GameId}nextplayer]%
   \index{game data >Xskaknextplayer=\textbackslash Xskak\meta{GameId}nextplayer}%
   \index{Xskaknextplayer=\textbackslash Xskak\meta{GameId}nextplayer}%
   «w» or «b». The player which will make the next move, set by
   \cs{newchessgame}, by \cs{resumechessgame} and during parsing.
   Identical to the color in {\texttt\textbackslash Xskak\meta{GameId}lastfen}.

\item[\textbackslash Xskak\meta{GameId}nextmovenr]%
   \index{game data >Xskaknextmovenr=\textbackslash Xskak\meta{GameId}nextmovenr}%
   \index{Xskaknextmovenr=\textbackslash Xskak\meta{GameId}nextmovenr}%
   The number of the next move, set during parsing. Identical to
   fullmove in {\texttt\textbackslash Xskak\meta{GameId}lastfen}.

\item[\textbackslash Xskak\meta{GameId}nextmoveid]%
   \index{game data >Xskaknextmoveid=\textbackslash Xskak\meta{GameId}nextmoveid}%
   \index{Xskaknextmoveid=\textbackslash Xskak\meta{GameId}nextmoveid}%
   Combination of \key{nextmovenr} and \key{nextplayer}.

\begin{LTXexample}[width=0.5\textwidth]
 \newchessgame[id=show,
               setfen=2K/k/7R,
               moveid=10b]
 \mainline{10... Ka8}

 \chessboard

 \xskakgetgame{nextplayer},
 \xskakgetgame{nextmovenr},
 \xskakgetgame{nextmoveid}
\end{LTXexample}


\item[\textbackslash Xskak\meta{GameId}diagramlist]%
   \index{game data >Xskakdiagramlist=\textbackslash Xskak\meta{GameId}diagramlist}%
   \index{Xskakdiagramlist=\textbackslash Xskak\meta{GameId}diagramlist}%
    A comma separated list of moves (a combination of movenr and
    player), (e.g. 3b,20w). Records the moves with an \$d or \$D
    after the move. See section \ref{sec:diagrams}.


\item[\textbackslash Xskak\meta{GameId}gameid]%
    \index{game data >Xskakgameid=\textbackslash Xskak\meta{GameId}gameid}%
    \index{Xskakgameid=\textbackslash Xskak\meta{GameId}gameid}%
    This command holds the game identification (the value is equal to the value of \meta{GameId}).

\begin{LTXexample}
 \newchessgame[id=game\arabic{section}]
 The id of the current game is \xskakgetgame{gameid}
\end{LTXexample}


\end{description}


\subsubsection{Optional game information: PGN-infos}

\DescribeMacro{\xskaknewpgninfo}%
The following commands store various optional information which are
normally stored in the so-called PGN-infos. You can set them in
\cs{newchessgame} and retrieve them later. New PGN-infos can be
defined with the \cs{xskaknewpgninfo}\oarg{default}\marg{name}. With
the exception of  «result» all currently defined commands are empty
by default.

\begin{description}
\item[\textbackslash Xskak\meta{GameId}result]%
    \index{PGN-info data >Xskakresult=\textbackslash Xskak\meta{GameId}result}%
    \index{Xskakresult=\textbackslash Xskak\meta{GameId}result}%
      Default is \verb+*+, value should be 1-0, or 1/2-1/2, 0-1 or
      something like that.

\item[\textbackslash Xskak\meta{GameId}white]%
    \index{PGN-info data >Xskakwhite=\textbackslash Xskak\meta{GameId}white}%
    \index{Xskakwhite=\textbackslash Xskak\meta{GameId}white}%
     The name of the white player.

\item[\textbackslash Xskak\meta{GameId}black]%
    \index{PGN-info data >Xskakblack=\textbackslash Xskak\meta{GameId}black}%
    \index{Xskakblack=\textbackslash Xskak\meta{GameId}black}%
     The name of the black player.


\item[\textbackslash Xskak\meta{GameId}whiteelo]%
    \index{PGN-info data >Xskakwhiteelo=\textbackslash Xskak\meta{GameId}whiteelo}%
    \index{Xskakwhiteelo=\textbackslash Xskak\meta{GameId}whiteelo}%
    The elo of the white player.

\item[\textbackslash Xskak\meta{GameId}blackelo]%
    \index{PGN-info data >Xskakblackelo=\textbackslash Xskak\meta{GameId}blackelo}%
    \index{Xskakblackelo=\textbackslash Xskak\meta{GameId}blackelo}%
    The elo of the black player.


\item[\textbackslash Xskak\meta{GameId}site]%
    \index{PGN-info data >Xskaksite=\textbackslash Xskak\meta{GameId}site}%
    \index{Xskaksite=\textbackslash Xskak\meta{GameId}site}%
   The site of the tournament.

\item[\textbackslash Xskak\meta{GameId}event]%
    \index{PGN-info data >Xskakevent=\textbackslash Xskak\meta{GameId}event}%
    \index{Xskakevent=\textbackslash Xskak\meta{GameId}event}%
The name of the tournament.



\item[\textbackslash Xskak\meta{GameId}date]%
    \index{PGN-info data >Xskakdate=\textbackslash Xskak\meta{GameId}date}%
    \index{Xskakdate=\textbackslash Xskak\meta{GameId}date}%
The date of the game.

\item[\textbackslash Xskak\meta{GameId}round]%
    \index{PGN-info data >Xskakround=\textbackslash Xskak\meta{GameId}round}%
    \index{Xskakround=\textbackslash Xskak\meta{GameId}round}%
The round of game.

\end{description}

\subsubsection{The move related data}\label{sec:movedata}

The move data are stored during the parsing of a game with
\cs{mainline} and/or \cs{hidemoves}. You can split the game in as
much pieces as you want but {\bfseries you must start a game with
\cs{newchessgame} and you must use the same game identification for
each piece} if you don't want to end with broken games.

All values depends on the parsing from \skaksty which in turn depends
heavily on the correct input: \skaksty knows a bit about the position
of the game (it needs it e.g. to find out where a piece came from)
but doesn't analyze it thoroughly. The main work is done by analyzing
the \emph{notation} e.g.  \skaksty will recognize a check by the
+-sign. Quite often \skaksty doesn't recognize illegal moves or wrong
input syntax (e.g. \texttt{0-0} for casting instead the correct
\texttt{O-O})  but fails some moves later or shows a wrong position:

\begin{LTXexample}
   \newchessgame
   \mainline{1.d4 e5  2.Bc4}
   \chessboard
\end{LTXexample}

\paragraph{Piece data}

The following types store the piece which was moved and the piece
that was perhaps captured. The variants differ in the storing of
black pieces and pawns:

\begin{description}
\item[pgnpiece]
    \index{Move data >pgnpiece}%
    \index{pgnpiece (move data type)}%
     K,Q,R,B or N. Is empty for pawn and
     castling moves.

\item[piecechar]
    \index{Move data >piecechar}%
    \index{piecechar (move data type)}%
     K,Q,R,B,N or p. Is empty for castling.


\item[piece]
    \index{Move data >piece}%
    \index{piece (move data type)}%
     K,Q,R,B,N or P or k,q,r,b,n or p (for black). Is empty for castling.


\item[pgnlostpiece]
    \index{Move data >pgnlostpiece}%
    \index{pgnlostpiece (move data type)}%
    In case of a capture move: K,Q,R,B, or N.

\item[lostpiecechar]
    \index{Move data >lostpiecechar}%
    \index{lostpiecechar (move data type)}%
     In case of a capture move: K,Q,R,B,N or p, empty
     else.


\item[lostpiece]
    \index{Move data >lostpiece}%
    \index{lostpiece (move data type)}%
     In case of a capture move it stores the lost piece as K,Q,R,B,N or P or
     k,q,r,b,n or p.

\end{description}


\paragraph{Move data}

\begin{description}
\item[movefrom]
    \index{Move data >movefrom}%
    \index{movefrom (move data type)}%
    The field were the piece came from. Two fields in case of castling moves e.g. «e1,h1».

\item[pgnmovefrom]
    \index{Move data >pgnmovefrom}%
    \index{pgnmovefrom (move data type)}%
    This contains the additional «movefrom» information in the PGN
    for ambiguous moves or pawn captures.
    E.g. In case of \wmove{exd5} the value is
    «e», in case of \wmove{Nbd2} it is «b».

\item[moveto]
    \index{Move data >moveto}%
    \index{moveto (move data type)}%
     The field were the piece move to. Two fields in case of castling moves e.g. «g1,f1».

\item[enpassant]
    \index{Move data >enpassant}%
    \index{enpassant (move data type)}%
    Sets the boolean \texttt{xskakboolenpassant} to true if the move
    was an enpassant capture.

\item[enpassantsquare]
    \index{Move data >enpassantsquare}%
    \index{enpassantsquare (move data type)}%
    A field, after e.g. e4 the value is e3. Indicates if the next
    pawn move can be an enpassant capture.

\begin{LTXexample}
 \newchessgame[restorefen=example]
 \mainline{1. a4 h5}
 \chessboard[lastmoveid,
             pgfstyle=border,
             color=red,
             markfield=\xskakget{enpassantsquare},
             moveid=1w, color=blue,
             markfield=\xskakget{enpassantsquare}]
\end{LTXexample}

\item[castling]
    \index{Move data >castling}%
    \index{castling (move data type)}%
    Sets the boolean \texttt{xskakboolcastling}, is true for short
    *and* long castling!


\item[longcastling]
    \index{Move data >longcastling}%
    \index{longcastling (move data type)}%
    Sets the boolean \texttt{xskakboollongcastling}.

\item[capture]
    \index{Move data >capture}%
    \index{capture (move data type)}%
    Sets the boolean \texttt{xskakboolcapture}.


\item[promotion]
    \index{Move data >promotion}%
    \index{promotion (move data type)}%
    Sets the boolean \texttt{xskakboolpromotion}.


\item[promotionpiece]
    \index{Move data >promotionpiece}%
    \index{promotionpiece (move data type)}%
    Q,R,B or N or q,r,b or n or empty.

\item[promotionpiecechar]
    \index{Move data >promotionpiecechar}%
    \index{promotionpiecechar (move data type)}%
    Q,R,B or N  or empty.

\begin{LTXexample}
 \newcommand\testpromotion{%
   \xskakget{promotion}%
   \ifthenelse{\boolean{xskakboolpromotion}}
    {A pawn promoted to
    \figsymbol{\xskakget{promotionpiecechar}}}
    {No promotion in this move}}
 \newchessgame[restorefen=example]
 \mainline{1. c8=Q+ Ke7}

 \xskakset{moveid=1w}
 \testpromotion

 \xskakset{moveid=1b}
 \testpromotion
\end{LTXexample}


\end{description}

\paragraph{Other data/combinations}

\begin{description}
\item[addpieces]
    \index{Move data >addpieces}%
    \index{addpieces (move data type)}%
       contains the (list of) piece(s) that should be put on the
       board.\\ E.g. \verb+{kg8,rf8}+ in case of black short castling.

\item[clearfields]
    \index{Move data >clearfields}%
    \index{clearfields (move data type)}%
       The list of fields that should be emptied on the board (equal
       to the movefrom field with the exception of enpassant moves).

\item[nextfen]
    \index{Move data >nextfen}%
    \index{nextfen (move data type)}%
       The fen position \emph{after} the current move.

\item[pastfen]
    \index{Move data >pastfen}%
    \index{pastfen (move data type)}%
       The fen position \emph{before} the current move (identical to
       the fen \emph{after} the previous move).

\item[san]
    \index{Move data >san}%
    \index{san (move data type)}%
      The  standard algebraic notation of the move. It use chess
      commands: E.g.
      \\\verb+\textsymfigsymbol{R}a2+ for \textsymfigsymbol{R}a2
      \\\verb+e\capturesymbol d5+ for e\capturesymbol d5,
      \\\verb+\castlingchar\castlinghyphen\castlingchar+ for
      \castlingchar\castlinghyphen\castlingchar\\
      It contains also the signs for chess and mate, but not the
      comments.

\item[lan]
    \index{Move data >lan}%
    \index{lan (move data type)}%
    Long algebraic notation of the move, similar to \texttt{san}.

\medskip
\begin{LTXexample}[varwidth=false,width=0.3\textwidth]
 \newchessgame
 \mainline{1.f3 e5 2. g4?? Qh4+#!}

 \xskakset{moveid=2w}%
 \xskakget{san}, \xskakget{lan}

 \xskakset{moveid=2b}%
  \xskakget{san}, \xskakget{lan}
\end{LTXexample}



\item[opennr]
    \index{Move data >opennr}%
    \index{opennr (move data type)}%
    This contains the code to print the number like \skaksty would do
    it if you start or continue a game  with this move. The style must be set independantly either with
    a \cs{styleX} of \packagename{skak} or -- if you use the new interface of
    \packagename{xskak} to set a style -- with \cs{xskakset} and \cs{mainlinestyle} or \cs{variationstyle}.

\medskip
\begin{LTXexample}[varwidth=false,width=0.3\textwidth]
 \newchessgame
 \mainline{1.e4 e5}

 \xskakset{moveid=1w}%
 \xskakget{opennr}\xskakget{lan},
 {\styleB\mainlinestyle
  \xskakget{opennr}\xskakget{lan}}


 \xskakset{moveid=1b}%
 \xskakget{opennr}\xskakget{lan}
 \xskakset{style=styleA,level=1}
 {\mainlinestyle
  \xskakget{opennr}\xskakget{lan}}

 \xskakset{style=UF}%return to my style
\end{LTXexample}

\item[comments]%
    \index{Move data >comments}%
    \index{comments (move data type)}%
   contains the short comments after a move (!,? \withattack{} etc)

\medskip
\begin{LTXexample}[varwidth=false,width=0.3\textwidth]
 \newchessgame
 \mainline{1.f3 e5 2. g4?? Qh4+#!}

 \xskakset{moveid=2w}%
 \xskakget{comments}

 \xskakset{moveid=2b}%
 \xskakget{comments}
\end{LTXexample}


\item[nag]
    \index{Move data >nag}%
    \index{nag (move data type)}
   NAG's and comments inputed with \cs{xskakcomment}.

\begin{LTXexample}[varwidth=false,width=0.3\textwidth]
 \newchessgame
 \mainline{1.f3 e5 2. g4
          $4 \xskakcomment{ Why this? } Qh4+#!}

 \xskakset{moveid=2w}%
 \xskakget{nag}
\end{LTXexample}

\end{description}

\subsection{Exporting the games}

\DescribeMacro{\xskakexportgames}%
With \cs{xskakexportgames}\keyoarg{} you can export games to a file.
As default the command exports the game data of the game with the
default \meta{GameId} to the file \texttt{xskakgames.xsk}\footnote{I
didn't find any other application that use this extension. So I think
it is quite safe.}.

The command accept the following keys:
\begin{description}
\item[\key{file=}\meta{string}]
   \index{xskakexportgames=\verb!*+\xskakexportgames+>file (key)}%
   \index{file (\texttt{\textbackslash xskakexportgames} key)}
   This changes the name of the file. The extension can not be changed.
\item[\key{games=\{}\meta{comma separated list of games}\texttt{\}}]
   \index{xskakexportgames=\verb!*+\xskakexportgames+>games (key)}%
   \index{games (\texttt{\textbackslash xskakexportgames} key)}%
   With this key you can chose the games that should be exported.
\end{description}


You can then input the games in other document and use there the game
data without have to do all the parsing with \cs{mainline} again
(which can be quite slow). The export file is also quite handy to
debug the game parsing and saving.

As an example I exported  in another document the following two games
which I will use later in this documentation:

\begin{lstlisting}
 \newchessgame[id=export,white=Deep Blue, black=Kasparov,result=1-0]
 \mainline{1.e4 c6 2.d4 d5 3.Nc3 dxe4 4.Nxe4 Nd7 5.Ng5 Ngf6 6.Bd3 e6
 7.N1f3 h6 8.Nxe6 Qe7 9.O-O fxe6 10.Bg6+ Kd8 \xskakcomment{Kasparov
 schüttelt kurz den Kopf} 11.Bf4 b5 12.a4 Bb7 13.Re1 Nd5 14.Bg3 Kc8
 15.axb5 cxb5 16.Qd3 Bc6 17.Bf5 exf5 18.Rxe7 Bxe7 19.c4}

 \newchessgame[white=none,black=none,id=exportb]
 \mainline{1. e4 e5}

 \xskakexportgames[games={export,exportb}]
\end{lstlisting}

\DescribeMacro{\xskakendgamedata}%
\DescribeMacro{\xskakcurrentgameid}%
The export consists more or less of a long list of definitions with
one notable exception: At the end of each game there is the command
\cs{xskakendgamedata}. As a default the command does nothing but if
you redefine it e.g. to \cs{printchessgame} (see section
\ref{sec:print}), it can be used to print all games during the input.
\cs{xskakcurrentgameid} holds the \meta{GameId} of the previous game.

\begin{LTXexample}
 \renewcommand\xskakendgamedata
  {\xskakset{id=\xskakcurrentgameid}%
   \xskakgetgame{white}--\xskakgetgame{black}:
   \printchessgame[style=styleB,level=1]
  \par\bigskip}
 \input{xskakgames.xsk}
\end{LTXexample}



\subsection{The inner technique: how the data are stored}

The game related data are stored in commands with names of
the following structure:\\[0.5ex]
%
\indent{\ttfamily Xskak\meta{GameId}\meta{type of Information}}\\[0.5ex]
%
If \meta{GameId} is a simple string of characters you can use the
command by putting a backslash before the name, in other case you
will have to use \cs{csname}\ldots\cs{endcsname}.

The move related data are stored in commands with names that
have the following structure:\\[0.5ex]
{\ttfamily Xskak.\meta{GameId}.\meta{movenr}.\meta{player}.\meta{type
of Information}}\\[0.5ex]
As the names contain points and a number it is not possible to
execute the commands  by putting a backslash before the name. If you
want to use them you will have to surround the name by
\cs{csname}\ldots\cs{endcsname}
 -- something that probably will not be necessary in normal cases as
there are  user commands that allows «normal» access.

During storing the data \meta{movenr} and \meta{player} are taken
from the current move as processed by \skaksty, that is
\verb+\mainline{15... Nf3}+ will store the data in \\{\ttfamily
Xskak.\meta{GameId}.15.b.\meta{type of Information}}. \\Renumbering
the moves during storing is not possible\footnote{But it would be
probably not difficult to implement. So if you need it badly, sent me
a feature request.}.

Example \ref{ex:internal} on page \pageref{ex:internal} shows an
example that use the internal commands. As you can see, the
\cs{csname}--\cs{endcsname}-input is quite powerful: You can
construct a command with the help of other (expandable) commands or
use the value of counters.

\minisec{Warning!} \verb+\csname whatever\endcsname+ doesn't give an
error if the command isn't defined. It simply defines \cs{whatever}
as \cs{relax} and then executes it. So it is easy to overlook typing
errors -- and then to wonder why one doesn't get the expected output.


\begin{LTXexample}[pos=t,
                   width=\textwidth,
                   float,caption=Using the internal commands,
                   label=ex:internal,
                   language=TeX,
                   texcs={mainline,figsymbol,xskakgameAlastmovenr,
                          xskakgameinitmovenr,xskakgameinitmovenr,
                          newchessgame,xskakgameinitmovenr,chessboard},
                   texcsstyle=*\color{blue},
                   texcs={[2]{ifthenelse,boolean,csname,endcsname,the,value,numexpr,relax}},
                   texcsstyle={*[2]{\color{red}}}]
 \begin{multicols}{2}
 \minisec{Game 1}\newchessgame[id=gameA]%
 \mainline{1. e4 e5 2. Nf3 Nc6 3. Bb5 Nf6 4. O-O Nxe4 5.Bxc6 bxc6}

 \minisec{Game 2}\newchessgame[id=game,restorefen=example,moveid=15w]%
 \mainline{15. c8=Q+ Ke7 16. Rd1}

 \minisec{Comments}\setcounter{enumi}{1}%
 In the first game, white moved at first a
 \figsymbol{\csname Xskak.gameA.\the\value{enumi}.w.piecechar\endcsname},
 then a \figsymbol{%
  \csname Xskak.gameA.\the\numexpr\the\value{enumi}+1\relax.w.piecechar\endcsname}
 and a \figsymbol{%
  \csname Xskak.gameA.\the\numexpr\the\value{enumi}+2\relax.w.piecechar\endcsname}.
 In his last move black took a
 \figsymbol{\csname Xskak.gameA.\XskakgameAlastmovenr.b.lostpiecechar\endcsname},
 white took a \figsymbol{\csname
 Xskak.gameA.5.w.lostpiecechar\endcsname}.

 In the second game the second move of white was
 \figsymbol{%
  \csname Xskak.game.\the\numexpr \Xskakgameinitmovenr +1\relax.w.piece\endcsname}%
 \csname Xskak.game.16.w.movefrom\endcsname--%
 \csname Xskak.game.16.w.moveto\endcsname.

 The position in the first game after the third move of black was:

 \chessboard[tinyboard, setfen=\csname Xskak.gameA.3.b.nextfen\endcsname]
 The fourth move of white was\csname
 Xskak.gameA.4.w.castling\endcsname
 \ifthenelse{\boolean{xskakboolcastling}}{}{not} a castling move.
 \end{multicols}
\end{LTXexample}

\clearpage


\section{Retrieving the stored information}

\subsection{Setting the variables}

\DescribeMacro{\xskakset}%
It should be clear that to retrieve the information you must tell
\packagename{xskak} the game and the move for which you want to get
data. This is done with \cs{xskakset}\keymarg{} which will set the
variables \meta{GameId} and/or \meta{MoveId} to the values listed in
the \keylist.

Some possible\footnote{More keys are described later.} keys are:
\begin{description}
\item[\key{id=}\meta{name}]
    \index{xskakset=\verb!*+\xskakset+>id (key)}%
    \index{id (\texttt{\textbackslash xskakset} key)}%
The key will set/change globally the active game identification, the
\meta{GameId}. This will also affect the storing of the following
moves! So if you change the \meta{GameId} in the middle of the
parsing of a game, don't forget to reset it to the old value before
continuing!


\item[\key{movenr=}\meta{number}, \key{player=}\meta{«w» or «b»},
\key{moveid=}\meta{number + «w» or «b»}]
    \index{xskakset=\verb!*+\xskakset+>movenr (key)}%
    \index{movenr (\texttt{\textbackslash xskakset} key)}%
    \index{xskakset=\verb!*+\xskakset+>player (key)}%
    \index{player (\texttt{\textbackslash xskakset} key)}%
    \index{xskakset=\verb!*+\xskakset+>moveid (key)}%
    \index{moveid (\texttt{\textbackslash xskakset} key)}%
    The keys will set the move number and the player. The values are global but
    the values of \key{movenr} and \key{player} are not expanded
    directly while the value of \key{moveid} is. The keys change only
    the internal commands that are used to retrieve information, the
    move counter and the boolean of \cs{mainline} are not affected by this keys.

\item[\key{stepmoveid=}\meta{number}]
    \index{xskakset=\verb!*+\xskakset+>stepmoveid (key)}%
    \index{stepmoveid (\texttt{\textbackslash xskakset} key)}%
\key{stepmoveid} picks up the current value of the move variables of
\packagename{xskak} and «adds» the given number of halfmoves. When no
number is given the value 1 is used and this gives the next move.
E.g. if the current move is \texttt{10w} and you use
\texttt{stepmoveid=3} then you get \texttt{11b}. Negative numbers can
be used, if this results in a move number smaller than 1, a warning
is given but not an error: This gives you the possibility e.g. in
loops to handle the case yourself.



\item[\key{lastmoveid=}\meta{GameId}]
    \index{xskakset=\verb!*+\xskakset+>lastmoveid (key)}%
    \index{lastmoveid (\texttt{\textbackslash xskakset} key)}%
    This key sets the \packagename{xskak} move identification
    commands
    to the number/color of the last move
    of the game \meta{GameId}. \meta{GameId} is optional.
    \texttt{lastmoveid=A} ist equivalent to
    \texttt{id=A,lastmoveid}.




\item[\key{tag=}\meta{name}, \key{ref\ldots=}\meta{name}]
     \index{xskakset=\verb!*+\xskakset+>tag (key)}%
     \index{tag (\texttt{\textbackslash xskakset} key)}%
     \index{xskakset=\verb!*+\xskakset+>refid (key)}%
     \index{refid (\texttt{\textbackslash xskakset} key)}%
     \index{xskakset=\verb!*+\xskakset+>refnextmoveid(key)}%
     \index{refnextmoveid (\texttt{\textbackslash xskakset} key)}%
     \index{xskakset=\verb!*+\xskakset+>refpastmoveid(key)}%
     \index{refpastmoveid (\texttt{\textbackslash xskakset} key)}%
     \index{xskakset=\verb!*+\xskakset+>refpastmovenr (key)}%
     \index{refpastmovenr (\texttt{\textbackslash xskakset} key)}%
     \index{xskakset=\verb!*+\xskakset+>refpastplayer (key)}%
     \index{refpastplayer (\texttt{\textbackslash xskakset} key)}%
     \index{xskakset=\verb!*+\xskakset+>refnextmovenr (key)}%
     \index{refnextmovenr (\texttt{\textbackslash xskakset} key)}%
     \index{xskakset=\verb!*+\xskakset+>refnextplayer (key)}%
     \index{refnextplayer (\texttt{\textbackslash xskakset} key)}%
     \index{xskakset=\verb!*+\xskakset+>refpast (key)}%
     \index{refpast (\texttt{\textbackslash xskakset} key)}%
     \index{xskakset=\verb!*+\xskakset+>refnext (key)}%
     \index{refnext (\texttt{\textbackslash xskakset} key)}%


\key{tag} saves the current values of \meta{GameId} and the move in
internal commands. They can then later be retrieved by the keys of
type \key{ref}.

For more information on the labeling and referencing see section
\ref{sec:labeling}.

\item[\key{defaultid=}\meta{name}]
     \index{xskakset=\verb!*+\xskakset+>defaultid (key)}%
     \index{defaultid (\texttt{\textbackslash xskakset} key)}%
     With this key you can change the default game identification.
     The change is global.

\begin{LTXexample}
 \setcounter{gamecounter}{4}
 \xskakset{defaultid=game\Alph{gamecounter}}

 \newchessgame[white=John]
 \mainline{1. e4 e5}

 \stepcounter{gamecounter}
 \newchessgame[white=Jack]
 \mainline{1. d4 d5}

 The white players were \XskakgameDwhite\ and \XskakgameEwhite.
\end{LTXexample}


\item[\key{defaultmovenr=}\meta{number},
      \key{defaultplayer=}\meta{«w» or «b»},]
\item[\key{defaultmoveid=}\meta{number}\meta{«w» or «b»}]
    \index{xskakset=\verb!*+\xskakset+>defaultmoveid (key)}%
    \index{defaultmoveid (\texttt{\textbackslash xskakset} key)}%
    \index{xskakset=\verb!*+\xskakset+>defaultmovenr (key)}%
    \index{defaultmovenr (\texttt{\textbackslash xskakset} key)}%
    \index{xskakset=\verb!*+\xskakset+>defaultplayer (key)}%
    \index{defaultplayer (\texttt{\textbackslash xskakset} key)}%
     With this keys you can change the default start move for the
     subsequent games. The change is global.

\item[\key{defaultfen=}\meta{FEN}]
    \index{xskakset=\verb!*+\xskakset+>defaultfen (key)}%
    \index{defaultfen (\texttt{\textbackslash xskakset} key)}%
    As a default \cs{newchessgame} sets the start position to a new
    game. You can change this default with this key. The change is
    global.

\end{description}



\subsection{Getting the game information}

\DescribeMacro{\xskakgetgame}%
With \cs{xskakgetgame}\marg{type} you can retrieve game data
for the currently active \meta{GameId} (which can be set/changed with
the commands \cs{newchessgame}, \cs{resumechessgame} and
\cs{xskakset}).

\begin{LTXexample}[width=0.5\textwidth]
 \newchessgame[id=A,white=John White,
               black=Jack Black,
               result=1--0]
 \minisec{\xskakgetgame{white}--
          \xskakgetgame{black}}
 \mainline{1.e4 e5  2.Bc4 Nc6 3. Qh5 Nf6 4.Qxf7#}
 \mbox{\xskakgetgame{result}}


 \newchessgame[id=B,
               white=Harry Red,
               black=Tom Green,
               result=0--1,
               setwhite={kd1},
               addblack={kd3,rc5}]
 \chessboard[setfen={\xskakgetgame{initfen}}]
 \minisec{\xskakgetgame{white} --
          \xskakgetgame{black}}
 \mainline{1. Ke1 Rf5 2. Kd1 Rf1#} \mbox{\xskakgetgame{result}}
\end{LTXexample}

It is also possible to get the stored information by simply calling
the command\\ \cs{xskak}\meta{gameid}\meta{type} (if \meta{gameid}
consists of letters).

\begin{LTXexample}[width=0.5\textwidth]
 \newchessgame[id=A,white=John White,
               black=Jack Black, result=1--0]
 \minisec{\XskakAwhite\ -- \XskakAblack}
 \mainline{1.e4 e5  2.Bc4 Nc6 3. Qh5 Nf6 4.Qxf7#}
 \mbox{\XskakAresult}
 \newchessgame[id=B,white=Harry Red,
               black=Tom Green, result=0--1]
 \minisec{\XskakBwhite\ -- \XskakBblack}
 \mainline{1.f3 e5  2.g4 Qh4#}
 \mbox{\XskakBresult}

 \minisec{List of Results}
 \begin{tabular}[t]{l@{\,--\,}lr}
 \XskakAwhite&\XskakAblack&\XskakAresult\\
 \XskakBwhite&\XskakBblack&\XskakBresult
 \end{tabular}

 \minisec{Final positions}
 \chessboard[tinyboard,lastmoveid=A,
          setfen=\xskakget{nextfen}]
 \chessboard[tinyboard,lastmoveid=B,
          setfen=\xskakget{nextfen}]
\end{LTXexample}

%If \meta{gameid} contains non-letter chars (e.g. a number) you must
%use\\
%\cs{csname}\texttt{~xskak}\meta{gameid}\meta{type}\cs{endcsname}.
%
%If the \meta{gameid} or the \meta{type} are stored in commands you
%can use this commands in \cs{csname}\texttt{\ldots}\cs{endcsname}:
%\cs{csname}\texttt{~xskak}\cs{mygameid}\cs{mygametype}\cs{endcsname}



\subsection{Getting the move information}

Move data are bit more difficult as they depend on more
variables.

At first, you can naturally use  a
\cs{csname}\texttt{\ldots}\cs{endcsname}:\\
\cs{csname}\texttt{~xskak}.\meta{gameid}.\meta{movenr}.\meta{player}.\meta{type}\cs{endcsname}.


But normally you should use a
\cs{xskakset}/\cs{xskakget}-combination%
\footnote{I had to implement the retrieving as a two-step-method to
 be able to use the retrieving command in the \keylist of
\cs{chessboard}. As the values are expanded during
 the processing complicated commands which use internally \cs{def}
 and \cs{setkeys} can not be used as values.}.


\DescribeMacro{\xskakget}%
\cs{xskakget}\marg{type} will pick up the current values of the
variables for the \meta{GameId} and the \meta{MoveId} and then
retrieve/print the content of  type \meta{type}.

    \index{Move data >movenr (virtual type)}%
    \index{movenr (virtual move data type)}%
    \index{Move data >moveid (virtual type)}%
    \index{moveid (virtual move data type)}%
    \index{Move data >player (virtual type)}%
    \index{player (virtual move data type)}%
\meta{type} can be any of the types described in \ref{sec:movedata}
and additionally the three virtual types \textsf{movenr},
\textsf{player} and \textsf{moveid} can be used which will output the
current move identification. This allows easier retrieving of the
current value of the internal variables e.g. in loops and tests.

\textbf{Attention:} The move variables are not changed by the parsing
command of \skaksty. And they are not stored like the other types --
you can't retrieve them with a
\cs{csname}\texttt{\ldots}\cs{endcsname} construction.

\begin{LTXexample}
\newchessgame[moveid=10b]
\mainline{10... e5}

The game started with move \xskakget{moveid}.

\xskakset{moveid=5w}
Now the moveid has value \xskakget{movenr}\xskakget{player}.
\end{LTXexample}



\subsection{Using the data with \texttt{\textbackslash chessboard}}

The move data can naturally be used with \cs{chessboard} e.g. to
highlight the last move with an arrow. The  \meta{MoveId} you want to
refer to can be set before the board with \cs{xskakset} but if this
were the sole possibility this would mean that you couldn't get data
from other moves. So I defined a bunch of new keys for
\cs{chessboard} (and changed some existing keys) which allows to
retrieve during building the board all available move datas. The most
important ones are keys that tells \cs{chessboard} the \meta{GameId}
and the \meta{MoveId} for which you want to retrieve datas:

\begin{description}
\item[\key{id=}\meta{name}]
    \index{chessboard=\verb!*+\chessboard+>id (key)}%
    \index{movenr (\texttt{\textbackslash chessboard} key)}%
    Like the other \key{id}-keys this one sets the variable  that
    identifies the game. But setting this key will \emph{not} affect
    the value of the game identification outside the board, any
    change is local to the board.

\textbf{Attention:} Setting this key will \emph{not} change the
default position shown by \cs{chessboard}. This is always the
«current» game.

\item[\key{movenr=}\meta{number}, \key{player=}\meta{«w» or «b»}, \key{moveid=}\meta{number + «w» or «b»}]
    \index{chessboard=\verb!*+\chessboard+>movenr (key)}%
    \index{movenr (\texttt{\textbackslash chessboard} key)}%
    \index{chessboard=\verb!*+\chessboard+>player (key)}%
    \index{player (\texttt{\textbackslash chessboard} key)}%
    \index{chessboard=\verb!*+\chessboard+>moveid (key)}%
    \index{moveid (\texttt{\textbackslash chessboard} key)}%
    This keys stores the number and player for which you want to retrieve
    move related information. Again any setting is local to the board.



\item[\key{stepmoveid=}\meta{number}]
    \index{xskakset=\verb!*+\chessboard+>stepmoveid (key)}%
    \index{stepmoveid (\texttt{\textbackslash chessboard} key)}%
Like the similar key of \cs{xskakset} \key{stepmoveid} picks up the
current value of the move variables of \packagename{xskak} and «adds»
the given number of halfmoves. But again when you use the key in the
argument of \cs{chessboard} the change is local to the board.

\textbf{Attention:} If you set the \meta{MoveId} with the above keys
to a value «outside» the game and then try to retrieve a data you
will probably get horrible errors.


\item[\key{lastmoveid=}\meta{gameid}]
   \index{chessboard=\verb!*+\chessboard+>lastmoveid (key)}%
   \index{lastmoveid (\texttt{\textbackslash chessboard} key)}
The usual case is probably that you want to use the (currently) last
move of a game. This can be done with this key. The (optional) value
lets you specify a game identification. \texttt{lastmoveid=A} ist
equivalent to \texttt{id=A,lastmoveid}. Again the setting is local to
the board.

\end{description}

    \index{xskakget=\verb!*+\xskakget+ (with \texttt{\textbackslash chessboard})}%
After you have set the move variables, you can use the command
\cs{xskakget}\marg{type} described above used without
problems\footnote{I hope but I haven't tested any possible usage.}
inside \cs{chessboard} as long as the value expands to something the
keys can handle (\key{markmove} e.g. will not like it if you ask it
to process a fen).

\begin{LTXexample}[width=0.4\textwidth]
 \newchessgame[id=A,white=John White,
               black=Jack Black, result=1--0]
 \mainline{1.e4 e5  2.Bc4 Nc6 3. Qh5 Nf6 4.Qxf7#}

 \newchessgame % a game between A and the board
 \mainline{1.e4}

 \setchessboard{shortenend=5pt,color=red}%
 \chessboard[lastmoveid=A,setfen=\xskakget{nextfen},
  pgfstyle=border,color=red,
  markfields={\xskakget{moveto},\xskakget{movefrom}},
  pgfstyle=straightmove,
  markmove=\xskakget{movefrom}-\xskakget{moveto}]
\end{LTXexample}


\begin{LTXexample}[width=0.4\textwidth,escapechar=/]
 \newchessgame[id=A]
 \mainline{1.e4 e5  2.Bc4 Nc6 3. Qh5 Nf6 4.Qxf7#}

 \xskakget{moveid}% to show /\meta{MoveId} is still 1w/

 \setchessboard{shortenend=5pt,color=red}%
 \chessboard[lastmoveid=A,setfen=\xskakget{nextfen},
  moveid=1w,
  pgfstyle=straightmove,
  markmove=\xskakget{movefrom}-\xskakget{moveto},
  stepmoveid=2,color=green,
  markmove=\xskakget{movefrom}-\xskakget{moveto},
  stepmoveid=2,color=blue,
  markmove=\xskakget{movefrom}-\xskakget{moveto},
  stepmoveid=2,color=yellow,
  markmove=\xskakget{movefrom}-\xskakget{moveto}]

\xskakget{moveid}% to show that /\meta{MoveId} is again 1w/
\end{LTXexample}


\subsection{Labeling a game}\label{sec:labeling}

It is a bit tedious to have to remember the number and the color of
the moves when you want to start or to continue a game, or for which
you want to draw a diagram or retrieve another move information. So
there is a sort of labeling system. In difference to the labeling in
normal \LaTeX\ it doesn't write the values to auxiliary files. That
means that you can't reference labels which come later in the
document (at my opinion this would lead to chaos anyway).



To set a label you use the command \cs{xskakset} with the key
\key{tag}\footnote{The name «label» is used by \cs{chessboard} to set
the labels of the boards.}.%
   \index{xskakset=\verb!*+\xskakset+>tag (key)}%
   \index{tag (\texttt{\textbackslash xskakset} key)}
It saves the current value of the move counter (the
\packagename{skak} counter!), the value of the (\packagename{skak}!)
boolean that holds the player and the \meta{GameId}. If you haven't
started any game yet (more precisely: if the move counter has value
0) then  \xskak will issue a warning and do nothing.

You can set the tag between two \cs{mainline} (or \cs{hidemoves})
commands, but it works also if you put them directly in the argument
of \cs{mainline}:

\begin{LTXexample}
 \newchessgame%
 \mainline{1. e4 e5\xskakset{tag=A}
           2. Nf3\xskakset{tag=B} Nc6
           3. Bb5 Nf6
           4. O-O Nxe4
           5.Bxc6 bxc6}

\end{LTXexample}


The saved values can be retrieved with various keys of type \key{ref}
(e.g. \key{refpastmoveid}, \key{refnextplayer}), or by the command
\cs{xskakget}.

The label is set in most cases between two moves. There probably will
be cases when you need the \meta{MoveId} of the move \emph{before}
the tag e.g. to show the position or to comment on the last move, and
there will be cases when you will need the \meta{MoveId} of the move
\emph{after} the tag e.g. to continue the game. So there are ref-keys
for both cases.

\subsubsection{Retrieving the tagged values with \texttt{\textbackslash
xskakget}}

Retrieving the \meta{GameId} and/or \meta{MoveId} values stored by a
\key{tag} is again a more-than-one-step procedure: At first you must
tell \packagename{xskak} the tag name you want to use, and then use a
\cs{xskakget}\marg{list of ref types}, where \meta{ref type} is one
of the following keywords: {\ttfamily refid, refpastmoveid,
refnextmoveid, refpastmovenr, refnextmovenr, refpastplayer,
refnextplayer}.


    \index{resumechessgame=\verb!*+\resumechessgame+>reftag (key)}%
    \index{reftag (\texttt{\textbackslash resumechessgame} key)}%
    \index{newchessgame=\verb!*+\newchessgame+>reftag (key)}%
    \index{reftag (\texttt{\textbackslash newchessgame} key)}%
    \index{xskakset=\verb!*+\xskakset+>reftag (key)}%
    \index{reftag (\texttt{\textbackslash xskakset} key)}%
    \index{xskakloop=\verb!*+\xskakloop+>reftag (key)}%
    \index{reftag (\texttt{\textbackslash xskakloop} key)}%
    \index{printchessgame=\verb!*+\xskakloop+>reftag (key)}%
    \index{reftag (\texttt{\textbackslash printchessgame} key)}%
    \index{chessboard=\verb!*+\chessboard+>reftag (key)}%
    \index{reftag (\texttt{\textbackslash chessboard} key)}%
Setting the tag name is done with the key \key{reftag}, which you can
use with all commands that accept the key \key{id}. If you use the
key in the optional argument of \cs{chessboard} it will change the
tag name locally for the board. In all other cases the tag name is
set globally.

\begin{LTXexample}[escapechar=/]
 \newchessgame[id=new] % to set a new id
 \chessboard[
   reftag=B,%from previous example
   id=\xskakget{refid},
    %retrieves and sets the /\meta{Gameid}/ from tag B
   moveid=\xskakget{refnextmoveid},
    %retrieves and sets the /\meta{Moveid}/ from tag B
   setfen=\xskakget{pastfen}]
\end{LTXexample}


\subsubsection{The \key{ref}-keys}

As the more-step-procedure to use the tags is a bit longuish there
exist keys that shorten the procedure. This keys too work in the
arguments of all commands that can also handle the correspondant keys
\key{id}, \key{moveid} etc. The effect will be identical to setting
the values directly: Keys used in the optional argument of
\cs{chessboard} will set their values only local to the board, in the
other cases the values are set globally.

\begin{description}
\item[\key{refid=}\meta{tag}]
    \index{resumechessgame=\verb!*+\resumechessgame+>refid (key)}%
    \index{refid (\texttt{\textbackslash resumechessgame} key)}%
    \index{newchessgame=\verb!*+\newchessgame+>refid (key)}%
    \index{refid (\texttt{\textbackslash newchessgame} key)}%
    \index{xskakset=\verb!*+\xskakset+>refid (key)}%
    \index{refid (\texttt{\textbackslash xskakset} key)}%
    \index{xskakloop=\verb!*+\xskakloop+>refid (key)}%
    \index{refid (\texttt{\textbackslash xskakloop} key)}%
    \index{chessboard=\verb!*+\chessboard+>refid (key)}%
    \index{refid (\texttt{\textbackslash chessboard} key)}%

This key will set the \meta{GameId} to the value stored in the tag.
It is a shortcut for\\
\key{reftag=}\meta{tag},\key{id=}\cs{xskakget}verb+{refid}+


\item[\key{refpastmoveid=}\meta{tag}, \key{refpastmovenr=}\meta{tag}, \key{refpastplayer=}\meta{tag} ]
    \index{resumechessgame=\verb!*+\resumechessgame+>refpastmoveid (key)}%
    \index{refpastmoveid (\texttt{\textbackslash resumechessgame} key)}%
    \index{resumechessgame=\verb!*+\resumechessgame+>refpastmovenr (key)}%
    \index{refpastmovenr (\texttt{\textbackslash resumechessgame} key)}%
    \index{resumechessgame=\verb!*+\resumechessgame+>refpastplayer (key)}%
    \index{refpastplayer (\texttt{\textbackslash resumechessgame} key)}%
  %
    \index{newchessgame=\verb!*+\newchessgame+>refpastmovenr (key)}%
    \index{refpastmovenr (\texttt{\textbackslash newchessgame} key)}%
    \index{newchessgame=\verb!*+\newchessgame+>refpastmoveid (key)}%
    \index{refpastmoveid (\texttt{\textbackslash newchessgame} key)}%
    \index{newchessgame=\verb!*+\newchessgame+>refpastplayer (key)}%
    \index{refpastplayer (\texttt{\textbackslash newchessgame} key)}%
    %
    \index{xskakset=\verb!*+\xskakset+>refpastmoveid (key)}%
    \index{refpastmoveid (\texttt{\textbackslash xskakset} key)}%
    \index{xskakset=\verb!*+\xskakset+>refpastmovenr (key)}%
    \index{refpastmovenr (\texttt{\textbackslash xskakset} key)}%
    \index{xskakset=\verb!*+\xskakset+>refpastplayer (key)}%
    \index{refpastplayer (\texttt{\textbackslash xskakset} key)}%
    %
    \index{chessboard=\verb!*+\chessboard+>refpastmoveid (key)}%
    \index{refpastmoveid (\texttt{\textbackslash chessboard} key)}%
    \index{chessboard=\verb!*+\chessboard+>refpastmovenr (key)}%
    \index{refpastmovenr (\texttt{\textbackslash chessboard} key)}%
    \index{chessboard=\verb!*+\chessboard+>refpastplayer (key)}%
    \index{refpastplayer (\texttt{\textbackslash chessboard} key)}%


This keys will set the respective move information to the value of
the move before the tag. If the tag is before the move 1w, then the
value 0b is used.

\item[\key{refnextmoveid=}\meta{tag}, \key{refnextmovenr=}\meta{tag}, \key{refnextplayer=}\meta{tag} ]
    \index{resumechessgame=\verb!*+\resumechessgame+>refnextmoveid (key)}%
    \index{refnextmoveid (\texttt{\textbackslash resumechessgame} key)}%
    \index{resumechessgame=\verb!*+\resumechessgame+>refnextmovenr (key)}%
    \index{refnextmovenr (\texttt{\textbackslash resumechessgame} key)}%
    \index{resumechessgame=\verb!*+\resumechessgame+>refnextplayer (key)}%
    \index{refnextplayer (\texttt{\textbackslash resumechessgame} key)}%
  %
    \index{newchessgame=\verb!*+\newchessgame+>refnextmovenr (key)}%
    \index{refnextmovenr (\texttt{\textbackslash newchessgame} key)}%
    \index{newchessgame=\verb!*+\newchessgame+>refnextmoveid (key)}%
    \index{refnextmoveid (\texttt{\textbackslash newchessgame} key)}%
    \index{newchessgame=\verb!*+\newchessgame+>refnextplayer (key)}%
    \index{refnextplayer (\texttt{\textbackslash newchessgame} key)}%
    %
    \index{xskakset=\verb!*+\xskakset+>refnextmoveid (key)}%
    \index{refnextmoveid (\texttt{\textbackslash xskakset} key)}%
    \index{xskakset=\verb!*+\xskakset+>refnextmovenr (key)}%
    \index{refnextmovenr (\texttt{\textbackslash xskakset} key)}%
    \index{xskakset=\verb!*+\xskakset+>refnextplayer (key)}%
    \index{refnextplayer (\texttt{\textbackslash xskakset} key)}%
    %
    \index{chessboard=\verb!*+\chessboard+>refnextmoveid (key)}%
    \index{refnextmoveid (\texttt{\textbackslash chessboard} key)}%
    \index{chessboard=\verb!*+\chessboard+>refnextmovenr (key)}%
    \index{refnextmovenr (\texttt{\textbackslash chessboard} key)}%
    \index{chessboard=\verb!*+\chessboard+>refnextplayer (key)}%
    \index{refnextplayer (\texttt{\textbackslash chessboard} key)}%


This keys will set the respective move information to the value of
the move after the tag.

\item[\key{refpast=}\meta{tag}, \key{refnext=}\meta{tag}]
    \index{resumechessgame=\verb!*+\resumechessgame+>refpast (key)}%
    \index{refpast (\texttt{\textbackslash resumechessgame} key)}%
    \index{newchessgame=\verb!*+\newchessgame+>refpast (key)}%
    \index{refpast (\texttt{\textbackslash newchessgame} key)}%
    \index{xskakset=\verb!*+\xskakset+>refpast (key)}%
    \index{refpast (\texttt{\textbackslash xskakset} key)}%
    \index{resumechessgame=\verb!*+\resumechessgame+>refnext (key)}%
    \index{refnext (\texttt{\textbackslash resumechessgame} key)}%
    \index{newchessgame=\verb!*+\newchessgame+>refnext (key)}%
    \index{refnext (\texttt{\textbackslash newchessgame} key)}%
    \index{xskakset=\verb!*+\xskakset+>refnext (key)}%
    \index{refnext (\texttt{\textbackslash xskakset} key)}%
    \index{chessboard=\verb!*+\chessboard+>refpast (key)}%
    \index{refpast (\texttt{\textbackslash chessboard} key)}%
    \index{chessboard=\verb!*+\chessboard+>refnext (key)}%
    \index{refnext (\texttt{\textbackslash chessboard} key)}%


This keys will set the \meta{GameId} and the \meta{MoveId} in one go.
E.g. \key{refpast} is a shortcut for\\
\key{reftag=}\meta{tag},\key{id=}\cs{xskakget}verb+{refid}+,\key{moveid=}\cs{xskakget}verb+{refpastmoveid}+




\item[\key{reffen=}\meta{tag}]
    \index{chessboard=\verb!*+\chessboard+>reffen (key)}%
    \index{reffen (\texttt{\textbackslash chessboard} key)}%
    \index{newchessgame=\verb!*+\newchessgame+>reffen (key)}%
    \index{reffen (\texttt{\textbackslash newchessgame} key)}%

This is a \cs{chessboard} key. It will add the position at the tag on
the board, its effect is  identical to using the key {\ttfamily
addfen =}\meta{Position at tag} described in the documentation of
\pchessboard. So be aware that it really only places  pieces on the
board. It will not change the \meta{GameId} nor the \meta{MoveId} or
the move counter of \packagename{skak}. So it is quite possible to
use a position where black should move next to start a game with
white to move.


\begin{LTXexample}
 \newchessgame[id=new]% to set a new id
 \chessboard[reffen=B]
\end{LTXexample}




\begin{LTXexample}
 \newchessgame[id=A]%
 \mainline{1.e4 e5 2. Nf3\xskakset{tag=C} Nc6
 3. Bb5 Nf6 4. O-O Nxe4 5.Bxc6 bxc6}


 \newchessgame[id=B,reffen=C]
 \mainline{1. Nc3}

 \chessboard

 \chessboard[reffen=C,
             markstyle=circle,
             color=red,
             %from game B:
             markfield=\xskakget{movefrom},
             refpast=C,
             color=green,
             %from game A:
             markfield=\xskakget{movefrom}]
\end{LTXexample}
\end{description}

\section{Looping}

Looping with the \cs{whiledo} command of the package
\packagename{ifthen} is not very difficult. But to adjust to the
special numbering of chessmoves (1w, 1b, 2w, 2b, 3w \ldots) one needs
some special commands to step through the moves and test for the end
of the loop.

\DescribeMacro{\xskaktestmoveid}%
Stepping can be done with the key \key{stepmoveid} described
earlier.\\ \cs{xskaktestmoveid}\marg{movenr}\marg{player} is a test
(defined with the help of package \packagename{xifthen}) which you
can use in the first argument of \cs{whiledo}. It will give true if
the move is in the current game.

\begin{LTXexample}
 \newchessgame[id=A]
 \mainline{1. e4 e5 2. Nf3 Nf6}

 \newchessgame[id=B]
 (a second game)

 Reprinting the first game:
 \xskakset{id=A,
        moveid=\xskakgetgame{initmoveid}}
 \whiledo{\xskaktestmoveid
   {\xskakget{movenr}}{\xskakget{player}}}
  {\ifthenelse%
    {\equal{\xskakget{player}}{w}}
    { \xskakget{movenr}.\,}
    { }%
    \xskakget{lan}%
    \xskakset{stepmoveid}}

 Reprinting only the black moves:
 \xskakset{%
   id=A,
   moveid=\xskakgetgame{initmoveid},stepmoveid}
 \whiledo{\xskaktestmoveid
    {\xskakget{movenr}}{\xskakget{player}}}
   { \xskakget{lan}%
    \xskakset{stepmoveid=2}}
\end{LTXexample}


\DescribeMacro{\xskakloop}%
\cs{xskakloop}\keyoarg\marg{code} is a wrapper command for such a
\cs{whiledo} loop.

The possible keys are:

\begin{description}
\item[\key{id=}\meta{string}]
   \index{xskakloop=\verb!*+\xskakloop+>id (key)}%
   \index{id (\texttt{\textbackslash xskakloop} key)}%
   Which sets the game identification. The setting is global.

\item[\key{initmovenr=}\meta{number}, \key{initplayer=}\meta{«w» or «b»},
\key{initmoveid=}\meta{number + «w» or «b»}]
    \index{xskakloop=\verb!*+\xskakloop+>initmovenr (key)}%
    \index{initmovenr (\texttt{\textbackslash xskakloop} key)}%
    \index{xskakloop=\verb!*+\xskakloop+>initplayer (key)}%
    \index{initplayer (\texttt{\textbackslash xskakloop} key)}%
    \index{xskakloop=\verb!*+\xskakloop+>initmoveid (key)}%
    \index{initmoveid (\texttt{\textbackslash xskakloop} key)}%
   This sets the move the loop should start with. Default is the
   first move of the game.

\item[\key{stopmovenr =}\meta{number}, \key{stopplayer =}\meta{«w» or «b»},
\key{stopmoveid=}\meta{number + «w» or «b»}]
    \index{xskakloop=\verb!*+\xskakloop+>stopmovenr (key)}%
    \index{stopmovenr (\texttt{\textbackslash xskakloop} key)}%
    \index{xskakloop=\verb!*+\xskakloop+>stopplayer (key)}%
    \index{stopplayer (\texttt{\textbackslash xskakloop} key)}%
    \index{xskakloop=\verb!*+\xskakloop+>stopmoveid (key)}%
    \index{stopmoveid (\texttt{\textbackslash xskakloop} key)}%
   This sets the move where the loop should stop. Default is the
   last move of the game.

\item[\key{step=}\meta{number}]
   \index{xskakloop=\verb!*+\xskakloop+>step (key)}%
   \index{step (\texttt{\textbackslash xskakloop} key)}%
   This sets the length of the steps of the loop. The number gives the number of
   halfmoves. Default is one.


\item[\key{showlast=}\meta{true|false}]
   \index{xskakloop=\verb!*+\xskakloop+>showlast (key)}%
   \index{showlast (\texttt{\textbackslash xskakloop} key)}%
   When you are using the key \key{step} with a number unequal to
   one, it is possible that the last move (possibly set by
   \key{stopmoveid}) is left out. With the key \key{showlast} you can force \cs{xskakloop} to process
   the code for the last move.
\end{description}


\begin{LTXexample}[pos=t,width=\textwidth]
 \newchessgame
 \mainline{%
 1.d4 Nf6 2.c4 g6 3.Nc3 Bg7 4.e4 d6 5.Nge2 O-O 6.Ng3 c6 7.Be2 a6 8.a4
 a5 9.h4 h5 10.Be3 Na6 11.f3 e5 12.d5 Nd7 13.Nf1 Ndc5 14.Nd2 Qb6
 15.Qb1 Nb4 16.Nb3 Ncd3+ 17.Kd2 Qxe3+}

 \xskakloop[step=6,showlast]{%
  \begin{tabular}{c}
   \chessboard[tinyboard,
               setfen=\xskakget{nextfen}]
   \\
   \xskakget{opennr}\xskakget{lan}%
  \end{tabular}\quad}%
\end{LTXexample}

With the help of the package \packagename{animate} and \cs{xskakloop}
you can make animated boards:\mynobreakpar
\begingroup
 \setchessboard{boardfontsize=0.8cm,shortenend=5pt, margin=false}%
 \setboardfontencoding{LSBC4}%
 \newskaklanguage{german}{KDTLSB}\skaklanguage[german]%
 %
 \newcommand\getmovestyle[1]{%
  \ifthenelse
    {\equal{#1}{N}}%knight move
    {\def\mymovestyle{[clockwise=false,style=knight]curvemove}}
    {\ifthenelse
      {\equal{#1}{}}% castling
      {\def\mymovestyle{curvemove}}
      {\def\mymovestyle{straightmove}}}}%
 %
 \newchessgame[white=Paethz,black=Dirr,result=0-1,id=anim]%
 \hidemoves{%
 1.d4 Sf6 2.c4 g6 3.Sc3 Lg7 4.e4 d6 5.Sge2 O-O
 6.Sg3 c6 7.Le2 a6 8.a4 a5 9.h4 h5 10.Le3 Sa6
 11.f3 e5 12.d5 Sd7 13.Sf1 Sdc5 14.Sd2 Db6
 15.Db1 Sb4 16.Sb3 Scd3+ 17.Kd2 Dxe3+}%
 %
 \unitlength0.8cm\relax
 \begin{animateinline}[autoplay,loop,controls]{0.5}%
 \begin{picture}(11,11)
 \put(1.5,1.5){\chessboard[setfen=\xskakgetgame{initfen}]}%
 \end{picture}
 \newframe
 \xskakloop{%
  \getmovestyle{\xskakget{piecechar}}%
  \begin{picture}(11,11)
  \put(1.5,1.5){%
   \chessboard[pgfstyle=\mymovestyle, color=blue,
               pgfshortenend=0.3em,arrow=to,
               markmoves=\xskakget{move},
               coloremph,piececolor=blue,
               emphfields={\xskakget{moveto}},
               setfen=\xskakget{nextfen}]}%
  \end{picture}%
   \newframe}%
 \begin{picture}(11,11)
 \put(1.5,1.5){\chessboard[setfen=\xskakgetgame{lastfen}]}%
 \end{picture}
 \end{animateinline}
\endgroup

\bigskip
The  board was made with the following code:

\begin{lstlisting}
 \setchessboard{boardfontsize=0.8cm,shortenend=5pt, margin=false}%
 \setboardfontencoding{LSBC4}%
 \newskaklanguage{german}{KDTLSB}\skaklanguage[german]%
 %
 \newcommand\getmovestyle[1]{%
  \ifthenelse
    {\equal{#1}{N}}%knight move
    {\def\mymovestyle{[clockwise=false,style=knight]curvemove}}
    {\ifthenelse
      {\equal{#1}{}}% castling
      {\def\mymovestyle{curvemove}}
      {\def\mymovestyle{straightmove}}}}%
 %
 \newchessgame[white=Paethz,black=Dirr,result=0-1,id=anim]%
 \hidemoves{%
 1.d4 Sf6 2.c4 g6 3.Sc3 Lg7 4.e4 d6 5.Sge2 O-O
 6.Sg3 c6 7.Le2 a6 8.a4 a5 9.h4 h5 10.Le3 Sa6
 11.f3 e5 12.d5 Sd7 13.Sf1 Sdc5 14.Sd2 Db6
 15.Db1 Sb4 16.Sb3 Scd3+ 17.Kd2 Dxe3+}%
 %
 \unitlength0.8cm\relax
 \begin{animateinline}[autoplay,loop,controls]{0.5}%
 \begin{picture}(11,11)
 \put(1.5,1.5){\chessboard[setfen=\xskakgetgame{initfen}]}%
 \end{picture}
 \newframe
 \xskakloop{%
  \getmovestyle{\xskakget{piecechar}}%
  \begin{picture}(11,11)
  \put(1.5,1.5){%
   \chessboard[pgfstyle=\mymovestyle, color=blue,
               pgfshortenend=0.3em,arrow=to,
               markmoves=\xskakget{move},
               coloremph,piececolor=blue,
               emphfields={\xskakget{moveto}},
               setfen=\xskakget{nextfen}]}%
  \end{picture}%
   \newframe}%
 \begin{picture}(11,11)
 \put(1.5,1.5){\chessboard[setfen=\xskakgetgame{lastfen}]}%
 \end{picture}
 \end{animateinline}
\end{lstlisting}

\section{Printing}
\subsection{\texttt{\textbackslash longmoves} revisitated}\index{longmoves=\verb!*+\longmoves+>}%



In the example on page \ref{ex:longmoves} I showed that the
\cs{longmoves} of \skaksty doesn't work correctly. This is due to the
asynchronous parsing and printing of the \packagename{skak}-engine
(\packagename{skak} -- like \TeX~-- has sometimes to look a bit
ahead). With \xskak the problem can be easily solved as it is now
possible to retrieve information about previous moves. So \xskak
redefines the internal commands to get \cs{longmoves} working.

\textbf{Attention!} The new \cs{longmoves} command works only
correctly if the internal engine parses the moves -- so {\em it
doesn't work with \cs{variation}}. So starting with version 1.2
\xskak forces \cs{variation} to use \cs{shortmoves}.


The redefinition of \cs{longmoves} is done by first defining a
printing command and then (re-)defining \cs{longmoves} such that the
internal \cs{printmove} now use the new command. The listing below
shows the code\footnote{The code doesn't show the complete truth. In
the real code \cs{longmoves} also has to change the printing mode for
\cs{printchessgame}. See \ref{sec:print}}. You can take it as a
template for your own special printing commands.
\cs{WhiteToMove}\marg{code for white}\marg{code for black} is a
command of \skaksty that executes the first argument if white is to
move and the second if its black turn. The main trick in the new
printing command is to go back one halfmove. The new printing command
defines only the printing format for the move itself as the number is
printed by \skaksty at another place.

\begin{lstlisting}
\def\xskak@do@printmove@algebraic{%
 \csname Xskak.\xskak@val@gameid.%
        \WhiteToMove{\the\numexpr\the\c@move-1\relax}{\the\c@move}.%
        \WhiteToMove{b}{w}.lan\endcsname
 \csname Xskak.\xskak@val@gameid.%
        \WhiteToMove{\the\numexpr\the\c@move-1\relax}{\the\c@move}.%
        \WhiteToMove{b}{w}.comments\endcsname{}}

\def\longmoves{%
 \let\printmove=\xskak@do@printmove@algebraic}
\end{lstlisting}

\DescribeMacro{\xskakenpassanttext}%
The new printing command inserts a test for enpassant moves. You can
suppress it by redefining \cs{xskakenpassantext}.



\begin{LTXexample}
 \longmoves
 \newchessgame
 \mainline{1.e4 e5 2.Nf3 Nf6}

 \newchessgame[restorefen=example]
 \mainline{1. a4 bxa3!}

 \renewcommand\xskakenpassanttext{}
 \newchessgame[restorefen=example]
 \mainline{1. a4 bxa3!}
\end{LTXexample}



\subsection{An user interface for styles}\label{sec:interface}

To format a game \skaksty puts during parsing commands before and
between the moves and the counter. You can see the names of this
commands in the following example (the style \key{test} is a simple
local style predefined by \packagename{xskak} which contains
different tests. In level 1 and 2 it contains style items to show
commands. The optional argument \texttt{[invar]} force \cs{variation}
to use level 2).

\begin{LTXexample}[pos=t,width=\textwidth,caption={The internal commands to format the output}]
 \newchessgame
 \xskakset{%
  style=test,
  level=1}
 \color{blue}

 \mainline{1.d4 Nf6 2.c4}

 \mainline{2... g6 3.Nc3 Bg7 4.e4}

 \mainline{4... O-O 5. Nf3 Re8}

 \mainline{6. Be3 e6}

 \variation[invar]{6... d6}
\end{LTXexample}

By judiciously (re-)definining the commands  \packagename{skak}
inserts one can get almost every chess style one want.
\packagename{skak} itself defines three examples that demonstrates
the possibilities. These styles   can be chosen by the commands
\cs{styleA}, \cs{styleB} and \cs{styleC}. The first two are inline
styles, the third uses for \cs{mainline} a tabbing environment and
switches to an inline style for \cs{variation}.  But this styles are
really no more than examples. There is no user interface to change
and define styles. And due to some dependencies between the
predefined styles (\cs{styleB} is not only a stand-alone style but it
is also used in the definition of \cs{styleC} for the variation
level) adapting the styles is not easy.


So I decided to define a real user interface to define styles and
switch from one to another. The user interface supports an unlimited
number of levels: Each style is a bundle of \emph{levels} where the
look on each level is describe by a \emph{style item}\index{style
item} and some optional  commands. Defining a style is therefore done
in two steps:

\begin{lstlisting}
 % Define items:
 \xskaknewstyleitem{A}{%
  beforenumber={},
  whiteopen={.\,},
  ...}
 \xskaknewstyleitem{B}{...}

 %Define style
 \xskaknewstyle[%
  level=1,styleitem=A,
    font=\bfseries, %add special command to level 1.
  level=2,styleitem=A,
  level=3,styleitem=B,
  ...]{mystyle}
 \end{lstlisting}


\textbf{Attention!} The (almost) indefinite number of levels comes at
a price: \cs{mainline}  and \cs{variation} no longer use
automatically different styles. If a \cs{variation} should switch to
a «variation style» you will have to use e.g. the option
\texttt{[invar]} mentioned above (but you can define your own
«variation» commands that do this switch). Please read section
\ref{sec:usestyles} for more information.



\subsubsection{Defining style items}

\DescribeMacro{\xskaknewstyleitem}%
With \cs{xskaknewstyleitem}\keyoarg\marg{item}  you define the style
item \meta{item}. You can use the following keys:


\begin{description}
\item[\meta{command name}=\meta{commands}]
   \index{xskaknewstyleitem=\verb!*+\xskaknewstyleitem+>opencommands (key)}%
   \index{opencommands (\texttt{\textbackslash xskaknewstyleitem} key)}%
   \index{xskaknewstyleitem=\verb!*+\xskaknewstyleitem+>closecommands (key)}%
   \index{closecommands (\texttt{\textbackslash xskaknewstyleitem} key)}%
   \index{xskaknewstyleitem=\verb!*+\xskaknewstyleitem+>beforenumber (key)}%
   \index{beforenumber (\texttt{\textbackslash xskaknewstyleitem} key)}%
   \index{xskaknewstyleitem=\verb!*+\xskaknewstyleitem+>whiteopen (key)}%
   \index{whiteopen (\texttt{\textbackslash xskaknewstyleitem} key)}%
   \index{xskaknewstyleitem=\verb!*+\xskaknewstyleitem+>beforewhite (key)}%
   \index{beforewhite (\texttt{\textbackslash xskaknewstyleitem} key)}%
   \index{xskaknewstyleitem=\verb!*+\xskaknewstyleitem+>afterwhite (key)}%
   \index{afterwhite (\texttt{\textbackslash xskaknewstyleitem} key)}%
   \index{xskaknewstyleitem=\verb!*+\xskaknewstyleitem+>blackopen (key)}%
   \index{blackopen (\texttt{\textbackslash xskaknewstyleitem} key)}%
   \index{xskaknewstyleitem=\verb!*+\xskaknewstyleitem+>beforeblack (key)}%
   \index{beforeblack (\texttt{\textbackslash xskaknewstyleitem} key)}%
   \index{xskaknewstyleitem=\verb!*+\xskaknewstyleitem+>afterblack (key)}%
   \index{afterblack (\texttt{\textbackslash xskaknewstyleitem} key)}%

{\raggedright where \meta{command name} is one of
\\\texttt{opencommands}, \texttt{closecommands},
\texttt{beforenumber}, \texttt{whiteopen}, \texttt{blackopen},
\texttt{beforewhite}, \texttt{beforeblack}, \texttt{afterwhite},
\texttt{afterblack}.\par}


There are endless possibilities for the value \meta{commands} -- as
there are endless possibilities to get more or less horrible errors.
So I will give only some general advices for the definitions of style
items:

\begin{itemize}
\item While processing \keylists\/ \packagename{xkeyval} strips spaces.
So if you want to set one of the commands to a space use
either commands like \verb+\ +, \verb+\space+ and \verb+~+, or
protect the space by enough layers of braces.

\item While processing \keylists\/ \packagename{xkeyval} strips also braces.
If you need some grouping use \cs{begingroup} and \cs{endgroup} or
protect the «real» braces by enough other braces.

\item Use if possible switches (like \cs{bfseries}) instead of commands with arguments.

\item Think about how you want to handle spaces at the start and the end of (bits of) a game
or a variation. An input like
 \verb+\mainline{1. e4}+\verb*+ \variation{1.e5}+ will already insert a
 space so avoid to insert another one through e.g. \key{opencommands}.


\item It is possible to open a group (with
\cs{begingroup}) or a environment in one command  and close it in
another command. If you do this check carefully that you have
balanced groups in all four combinations of white/black opening and
ending the game! Check also if everything works if you have only one
move.

\item You don't have to give a definition for all keys/commands.
Unused keys are set to empty.

\item If you want to force a style (or a style item) to use the long
or the short algebraic notation you can add
\cs{longmoves}/\hspace{0pt}\cs{shortmoves} to \key{opencommands}.
    \index{longmoves=\verb!*+\longmoves+>}%
    \index{shortmoves=\verb!*+\shortmoves+>}%
\end{itemize}


\item[\meta{beforeNAG}=\meta{commands}, \meta{beforecomment}=\meta{commands}]

\packagename{xskak} puts around each NAG and comment given by
\cs{xskakcomment} a group and before each NAG's and comment a
formatting command. This commands are named \cs{xskak@beforeNAG} and
\cs{xskak@beforecomment}. With the keys \key{beforeNAG} and
\key{beforecomment} you can change this commands and so to some
extend change the formatting of NAG and long comments (the styles of
short comments like + or ! are not affected!). The last command in
\meta{commands} can be a command that has one argument. NAG and the
content of the comment will then be used as this argument. This makes
is e.g. possible to gobble commands:

\begin{LTXexample}[varwidth=false,width=0.3\textwidth]
 \providecommand\gobble[1]{}
 \xskaknewstyleitem[%
   template=UF,%a predefined style item
   beforeNAG=\gobble,
   beforecomment=\gobble]{gobblecomments}

 \xskakaddtostyle[%
  level=gobble,
  styleitem=gobblecomments]{test}

 \variation[style=test,level=gobble]{1. e4! $1 \xskakcomment{expected} e5 $2}
\end{LTXexample}






\item[\key{template}=\meta{item}]
   \index{xskaknewstyleitem=\verb!*+\xskaknewstyleitem+>template (key)}%
   \index{template (\texttt{\textbackslash xskaknewstyleitem} key)}%

With this key you can reuse the definitions from another style item.
\end{description}


\subsubsection{Defining style}


\DescribeMacro{\xskaknewstyle}%
\DescribeMacro{\xskakaddtostyle}%
With \cs{xskaknewstyle}\keyoarg\marg{style}  you define the style
\meta{style}. \\With \cs{xskakaddtostyle}\keyoarg\marg{style} you can
add more levels (or overwrite existing ones) to a style. Both
commands can handle four keys: \key{level}, \key{styleitem},
\key{font} and \key{xfont}.


\begin{description}
\item[\key{level}=\meta{number}]
   \index{xskaknewstyle=\verb!*+\xskaknewstyle+>level (key)}%
   \index{level (\texttt{\textbackslash xskaknewstyle} key)}%

With this key you set the number of the level for which you want to
declared a «look» with the following keys. The «highest» level number
is 1. Normally it should contain the description of the style for the
main game. Variation styles should have larger numbers. But the
\meta{number} don't need to be a number, strings are also possible.


If you want to keep everything simple: Use only (positive) numbers
with levels. Then the levels have a well-defined order which is easy
to handle. Key \key{invar} will add 1 to a level number, key
\key{outvar} will substract 1 and if you set the level to a number
for which no style item as been declared \packagename{xskak} will
simply substract 1 from this number until it finds a level with a
style item it can use.

But you can also use level names with chars e.g. to define variants
for a specific variation level. So you could e.g. define levels
\texttt{1open}, \texttt{1close}, \texttt{1middle},
\texttt{1betteris}.

In this case it depends on your compiler how the  keys \key{invar}
and \key{outvar} and levels without declared style items are handled:

\begin{description}
\item[Case 1:] You have an old pdf\TeX\ where the (experimental) primitive
\cs{pdfmatch} is not defined (or you did use the package option
\texttt{nopdfmatch}).\mynobreakpar

Then the keys \key{invar} and \key{outvar} will not work when the
current level contains chars or other non-numerical tokens. They will
issue more or less bewildering errors.

Levels without a declared style item will issue errors too.

\item[Case 2:] You have a recent pdf\TeX\ where the (experimental) primitive
\cs{pdfmatch} is defined (and you didn't use the package option
\texttt{nopdfmatch}).\mynobreakpar

    \begin{description}
    \item[Case a)] The level name doesn't contain any number, e.g. \texttt{level=gobble}:\mynobreakpar

     Then you will get an error if you try to set the level and you haven't declared a style item for this level.

     If the current level is \texttt{gobble} then the keys \key{invar} and \key{outvar} will issue a warning
     and let the level unchanged.

    \item[Case b)] The level name do contain numbers, e.g. \texttt{level=a12b567}:\mynobreakpar

    Then \packagename{xskak} will use the following substituation if you try to set the level and there is no style
    item  declared:
    \packagename{xskak} will extract the
    first number from the name (in this case «12»), substract one
    from this number and try the name with the new number in it. If
    the first number is 1, level 1 is used (it is always defined). If
    the first number is 0 you will get an error.

    In the example \packagename{xskak} will try the levels in the following order until it finds one that it can use:
    {\ttfamily a12b567 $\rightarrow$ a11b567 $\rightarrow$ a10b567 $\rightarrow$ \ldots a2b567 $\rightarrow$ a1b567 $\rightarrow$ 1}
    \end{description}

\end{description}


\item[\key{styleitem}=\meta{item}]
   \index{xskaknewstyle=\verb!*+\xskaknewstyle+>styleitem (key)}%
   \index{styleitem (\texttt{\textbackslash xskaknewstyle} key)}%

With this key you set the style item used by the current level
\emph{and all following levels} until another \key{styleitem} is set.


\item[\key{font}=\meta{commands}]
   \index{xskaknewstyle=\verb!*+\xskaknewstyle+>font (key)}%
   \index{font (\texttt{\textbackslash xskaknewstyle} key)}%

With this key you can add  some specific commands to the current
level (\emph{and only to the current level}). While the key is meant
to add font switches to a level it can also be used to overwrite some
commands of a style item.



\begin{LTXexample}[varwidth=false,width=0.3\textwidth]
 \xskaknewstyleitem[%
  template=UF,%a predefined style item
  beforenumber=\color{red},%
  whiteopen=.\,\color{black},
  blackopen=\ldots\,\color{black}]{rednumbers}

 %% add to existing style test
 \xskakaddtostyle[%
  level=10,styleitem=rednumbers,
  level=12,
  font=\def\beforenumber{\color{green}}\bfseries]
   {test}


 \newchessgame
 \mainline[style=test,level=12]{1. Nf3 Nf6 2. d4}

 \variation[level=10]{2. Nc3}


 \mainline[level=11]{2... d5}
 \end{LTXexample}

\begin{LTXexample}[varwidth=false,width=0.3\textwidth]
\xskakset{style=test}%
 \xskaknewstyleitem[%
  template=UF,%a predefined style item
  beforeNAG=\color{blue},
  beforecomment=\mdseries]{mycomments}

 \xskakaddtostyle[level=13,styleitem=mycomments,
                 font=\bfseries]{test}

 \variation[level=13]{1. e4 $1 \xskakcomment{ as expected} e5}

 \makeatletter
 \xskakaddtostyle[%
   level=gobbleB,styleitem=gobblecomments,
   font=\renewcommand\xskak@beforeNAG{\color{red}}]%
   {test}
 \makeatother

 \variation[level=gobbleB]{1. e4! $1 \xskakcomment{ I expected it} e5 $2}
 \end{LTXexample}

\item[\key{xfont}=\meta{commands}]
   \index{xskaknewstyle=\verb!*+\xskaknewstyle+>xfont (key)}%
   \index{xfont (\texttt{\textbackslash xskaknewstyle} key)}%

With this key you can add  some specific commands to all levels with
the same first number in their name. It needs a recent pdf\TeX\ with
the command \cs{pdfmatch}.

As an example lets add three levels to the test style used above: the
first (main) level should use a bold font, the second \cs{mdseries}
and the third an italic font. For each level we will define a variant
which puts a \cs{betteris} (\betteris) before the first move.

The example also demonstrates how to retrieve the current level.
Please note the braces around some of the \cs{variations} which keep
the level change local.

\begin{LTXexample}[width=0.3\textwidth,varwidth=false,caption=«betteris» style,label=sty:betteris]
 \xskaknewstyleitem[%
  template=UF,
  opencommands=(\betteris,
  closecommands=)]{betteris}


 \xskakaddtostyle
 [level=20   ,styleitem=UF,xfont=\bfseries,
  level=20bis,styleitem=betteris,
  level=21   ,xfont=\mdseries,
  level=22   ,xfont=\itshape]{test}

 \xskakset{style=test}

 \variation[level=20]{1. e4 e5}   (level \xskakget{level})
 \variation[level=20bis]{1... e6} (level \xskakget{level})

 \variation[level=21]{1. e4 e5}   (level \xskakget{level})
 \variation[level=21bis]{1... e6} (level \xskakget{level})

 \variation[level=22]{1. e4 e5}   (level \xskakget{level})
 \variation[level=22bis]{1... e6} (level \xskakget{level})


 \variation[level=20]{1. e4 e5}
 {\variation[level=\xskakget{level}bis]{1... e6}
  \variation[invar]{1... e6}  \variation[invar]{1... e6}}

 \variation[invar]{1. e4 e5} %to level 21
 {\variation[level=\xskakget{level}bis]{1... e6}}

 \variation[invar]{1. e4 e5} %to level 22
 {\variation[level=\xskakget{level}bis]{1... e6}}

\end{LTXexample}


\end{description}


The previous examples demonstrate some important points to consider
when defining styles:

\begin{itemize}
\item You can define as much levels as you want.

\item The number of the levels don't have to be consecutive (they even need to be numbers). If there
is no definition for a level (in the first example level 11), the
style item of a previous level (in this case level 10) is used if
\packagename{xskak} is able to figure out what the previous level
could be.

\item While it is often sensible to store the settings for the main level
in level 1, and for the first variation level in level 2 and so on,
you can also chose other orders or use more than one level for each
logical variation level. If you e.g. want sometimes to add braces
through \key{opencommands} you could define variants $n$a, $n$b etc
of the involved variation level.



\item While it is easy to define style items and styles it is not
easy sort and bundle them in a sensible way.

\end{itemize}


\subsection{Using the styles}\label{sec:usestyles}

The names of the commands \cs{mainline} and \cs{variation} give the
impression that you should use the first for the main game and the
second for the variations. But at my opinion this impression is
wrong: You should use
 \begin{itemize}
 \item \cs{mainline} if you want to parse \emph{and} print chess moves,
 \item \cs{variation} if you want only to print chess moves,
 \item \cs{hidemoves} if you want only to parse moves and
 \item \cs{printchessgame} (described later) if you want to
  (re)print a previously parsed game.
 \end{itemize}

This means that each of the printing commands can be used for the
main game and for variations. And so it should be possible for each
of the printing commands to set and use every style level.

In consequence this means that if you use the new interface -- you
don't have to -- \cs{variation} will no longer switch to another
style.

You can set the style and the level either with \cs{xskakset} or in
the optional argument of the three printing commands. The keys you
can use in both cases are:



\begin{description}
\item[\key{style}=\meta{style name}]
    \index{xskakset=\verb!*+\xskakset+>style (key)}%
    \index{style (\texttt{\textbackslash xskakset} key)}%
    \index{mainline=\verb!*+\mainline+>style (key)}%
    \index{style (\texttt{\textbackslash mainline} key)}%
    \index{variation=\verb!*+\variation+>style (key)}%
    \index{style (\texttt{\textbackslash variation} key)}%
    \index{printchessgame=\verb!*+\printchessgame+>style (key)}%
    \index{style (\texttt{\textbackslash printchessgame} key)}%

This key sets the name of the style the following printing commands
should use. The setting is always local to the current group (even
with \cs{xskakset}\footnote{At first \cs{xskakset} did the settings
globally. I changed it after the third time I had to correct the
documentation because I forgot to reset a style after an example.}.


\item[\key{gstyle}=\meta{style name}]
    \index{xskakset=\verb!*+\xskakset+>gstyle (key)}%
    \index{gstyle (\texttt{\textbackslash xskakset} key)}%
    \index{mainline=\verb!*+\mainline+>gstyle (key)}%
    \index{gstyle (\texttt{\textbackslash mainline} key)}%
    \index{variation=\verb!*+\variation+>gstyle (key)}%
    \index{gstyle (\texttt{\textbackslash variation} key)}%
    \index{printchessgame=\verb!*+\printchessgame+>gstyle (key)}%
    \index{gstyle (\texttt{\textbackslash printchessgame} key)}%

This key sets the name of the style globally.


\item[\key{level}=\meta{number}]
    \index{xskakset=\verb!*+\xskakset+>level (key)}%
    \index{level (\texttt{\textbackslash xskakset} key)}%
    \index{mainline=\verb!*+\mainline+>level (key)}%
    \index{level (\texttt{\textbackslash mainline} key)}%
    \index{variation=\verb!*+\variation+>level (key)}%
    \index{level (\texttt{\textbackslash variation} key)}%
    \index{printchessgame=\verb!*+\printchessgame+>level (key)}%
    \index{level (\texttt{\textbackslash printchessgame} key)}%

This key sets the level the following printing commands should use.
The setting is always local to the current group.

\item[\key{glevel}=\meta{number}]
    \index{xskakset=\verb!*+\xskakset+>glevel (key)}%
    \index{glevel (\texttt{\textbackslash xskakset} key)}%
    \index{mainline=\verb!*+\mainline+>glevel (key)}%
    \index{glevel (\texttt{\textbackslash mainline} key)}%
    \index{variation=\verb!*+\variation+>glevel (key)}%
    \index{glevel (\texttt{\textbackslash variation} key)}%
    \index{printchessgame=\verb!*+\printchessgame+>glevel (key)}%
    \index{glevel (\texttt{\textbackslash printchessgame} key)}%

This key sets the level globally.

\item[\key{invar}]
    \index{xskakset=\verb!*+\xskakset+>invar (key)}%
    \index{invar (\texttt{\textbackslash xskakset} key)}%
    \index{mainline=\verb!*+\mainline+>invar (key)}%
    \index{invar (\texttt{\textbackslash mainline} key)}%
    \index{variation=\verb!*+\variation+>invar (key)}%
    \index{invar (\texttt{\textbackslash variation} key)}%
    \index{printchessgame=\verb!*+\printchessgame+>invar (key)}%
    \index{invar (\texttt{\textbackslash printchessgame} key)}%

This will add 1 to the first number in the current level (so it goes
towards «higher» (deeper) levels). Again the change is local to the
group. This key can break if the current value of the level is not a
number.

\item[\key{ginvar}]
    \index{xskakset=\verb!*+\xskakset+>ginvar (key)}%
    \index{ginvar (\texttt{\textbackslash xskakset} key)}%
    \index{mainline=\verb!*+\mainline+>ginvar (key)}%
    \index{ginvar (\texttt{\textbackslash mainline} key)}%
    \index{variation=\verb!*+\variation+>ginvar (key)}%
    \index{ginvar (\texttt{\textbackslash variation} key)}%
    \index{printchessgame=\verb!*+\printchessgame+>ginvar (key)}%
    \index{ginvar (\texttt{\textbackslash printchessgame} key)}%

This is the global variant of \key{invar}.

\item[\key{outvar}]
    \index{xskakset=\verb!*+\xskakset+>outvar (key)}%
    \index{outvar (\texttt{\textbackslash xskakset} key)}%
    \index{mainline=\verb!*+\mainline+>outvar (key)}%
    \index{outvar (\texttt{\textbackslash mainline} key)}%
    \index{variation=\verb!*+\variation+>outvar (key)}%
    \index{outvar (\texttt{\textbackslash variation} key)}%
    \index{printchessgame=\verb!*+\printchessgame+>outvar (key)}%
    \index{outvar (\texttt{\textbackslash printchessgame} key)}%

This will substract 1 to the first number in the current level (so it
goes towards level 1 as the main level). This key can break if the
current value of the level is not a number.

\item[\key{goutvar}]
    \index{xskakset=\verb!*+\xskakset+>goutvar (key)}%
    \index{goutvar (\texttt{\textbackslash xskakset} key)}%
    \index{mainline=\verb!*+\mainline+>goutvar (key)}%
    \index{goutvar (\texttt{\textbackslash mainline} key)}%
    \index{variation=\verb!*+\variation+>goutvar (key)}%
    \index{goutvar (\texttt{\textbackslash variation} key)}%
    \index{printchessgame=\verb!*+\printchessgame+>goutvar (key)}%
    \index{goutvar (\texttt{\textbackslash printchessgame} key)}%

This is the global variant of \key{outvar}.
\end{description}


\subsection{The predefined styles}

\Xskak predefines five styles: \texttt{styleA}, \texttt{styleB} and
\texttt{styleC} imitate the predefined styles of \packagename{skak}
with the addition that they change to italic in the third level.
Style \texttt{UF} is my style and style \texttt{test} is a large
container for all sorts of tests.

\begin{LTXexample}[width=0.4\textwidth,varwidth=false]
\newchessgame
 \mainline[style=styleC,level=1]{%
  1. e4 e6 2. d4 d5 3. e5 c5 4. c3 Qb6}
  (\variation[invar]{4... Ne7 5. Nf3 Nec6}
    (\variation[invar]{5... Nf5 6. h4 Be7
      7. Bd3 cxd4 8. cxd4 Nc6 9. Bxf5 exf5 10. Bg5 $14})
  \variation[outvar]{6. Bd3}
    (\variation[invar]{6. h4 Nd7 7. h5 f6 $1 $132}))
 \mainline[outvar,outvar]{ 5. Nf3 Bd7}

\end{LTXexample}


\subsection{The new \texorpdfstring{\texttt{xskak}}{xskak} printing
command}\label{sec:print}

\DescribeMacro{\printchessgame}%
As one of the aims of \packagename{xskak} is to separate the parsing
from the printing of a game there is naturally also a printing
command. With \cs{printchessgame}\keyoarg{} you can print a
previously parsed and saved game. Without any options
\cs{printchessgame} will print the game with the currently active
\meta{GameId} from start to end.



As keys you can use
\begin{description}
\item[\key{id}=\meta{GameId}, \key{refid}=\meta{tag}]
    \index{printchessgame=\verb!*+\printchessgame+>id (key)}%
    \index{id (\texttt{\textbackslash printchessgame} key)}%
    \index{printchessgame=\verb!*+\printchessgame+>refid (key)}%
    \index{refid (\texttt{\textbackslash printchessgame} key)}%

This keys change only for the printing the current \meta{GameId}.
Afterwards the old \meta{GameId} is restored.

\item[\key{initmovenr=}\meta{number}, \key{initplayer=}\meta{«w» or «b»}, \key{initmoveid=}\meta{number + «w» or «b»}]\index{printchessgame=\verb!*+\printchessgame+>initmovenr (key)}%
    \index{initmovenr (\texttt{\textbackslash printchessgame} key)}%
    \index{printchessgame=\verb!*+\printchessgame+>initplayer (key)}%
    \index{initplayer (\texttt{\textbackslash printchessgame} key)}%
    \index{printchessgame=\verb!*+\printchessgame+>initmoveid (key)}%
    \index{initmoveid (\texttt{\textbackslash printchessgame} key)}%
~\\ This keys sets the move the printing should start with. If the
move isn't in the game a warning is issued and the first move of the
game is used instead.

\item[\key{stopmovenr=}\meta{number}, \key{stopplayer=}\meta{«w» or «b»}, \key{stopmoveid=}\meta{number + «w» or «b»}]
    \index{printchessgame=\verb!*+\printchessgame+>stopmovenr (key)}%
    \index{stopmovenr (\texttt{\textbackslash printchessgame} key)}%
    \index{printchessgame=\verb!*+\printchessgame+>stopplayer (key)}%
    \index{stopplayer (\texttt{\textbackslash printchessgame} key)}%
    \index{printchessgame=\verb!*+\printchessgame+>stopmoveid (key)}%
    \index{stopmoveid (\texttt{\textbackslash printchessgame} key)}%
~\\ This keys sets the move the printing should end with. If the move
isn't in the game a warning is issued and the last move of the game
is used instead.

\item[\key{reftag}=\meta{tag}]
    \index{printchessgame=\verb!*+\printchessgame+>reftag (key)}%
    \index{reftag (\texttt{\textbackslash printchessgame} key)}%

This sets the tag name in case that you want to use tagged moves as
start or end of the printing. There are no keys like
\key{initrefpastmoveid} (I thought so long names are no longer
manageable), so you will have to use e.g. \texttt{reftag=}\meta{tag},
\texttt{initmovenr=}\cs{xskakget}\verb+{refpast}+.


\item[\key{style}, \key{level}, key{invar}  and \key{outvar}]%
    \index{printchessgame=\verb!*+\printchessgame+>level (key)}%
    \index{level (\texttt{\textbackslash printchessgame} key)}%
    \index{printchessgame=\verb!*+\printchessgame+>invar (key)}%
    \index{invar (\texttt{\textbackslash printchessgame} key)}%
    \index{printchessgame=\verb!*+\printchessgame+>outvar (key)}%
    \index{outvar (\texttt{\textbackslash printchessgame} key)}%
    \index{printchessgame=\verb!*+\printchessgame+>style (key)}%
    \index{style (\texttt{\textbackslash printchessgame} key)}%

\cs{printchessgame} use the same formatting commands as
\packagename{skak}, so the style can be change through the interface
defined and described in section \ref{sec:interface}.


To set the level and the style you can use the keys described in
section \ref{sec:usestyles} in the optional argument of
\cs{printchessgame}. Like with \cs{mainline} and \cs{variation} the
keys sets the values locally for the current group.

To print the actual moves, \cs{printchessgame} doesn't use the
command \cs{printmove} from \packagename{skaksty} but an internal
command named \cs{xskak@do@printmove}. So commands like
\cs{longmoves} must set this command appropriately.


\end{description}

\begin{LTXexample}
 \input{xskakgames.xsk}
 \printchessgame[style=UF,level=1,id=export,
                 initmoveid=7w,stopmoveid=12b]
\end{LTXexample}


\subsection{Game titles and chessboard captions}

There are no pre-made commands. You will have to define them with the
help of the stored game data/move data. For titles I advise you to
use a sort of sectioning command -- this will prevent page breaks
between the title and the game. Look e.g. at the definition of
\cs{minisec} in one of the classes from the KOMA-bundle. For
captions: look at the  package \packagename{caption}.




\section{PGN2LTX or How to sort the input}

While \packagename{skak}/\packagename{xskak} can handle games in
PGN-notation they can't handle PGN-files: neither the PGN-infos
(which is a small nuisances) nor the variations and the comments in
the game notation (which makes it tedious to print large games with a
lot of annotations).

There exists a small number of perl scripts and applications to help
to convert PGN to \LaTeX\ but none is really satisfactory. One of the
problems that prevents better solutions is that there is no clear
description what should be the result of a conversion from PGN to
\LaTeX. In this section I will make some remarks about what I think
would be a good output from such a converter for
\packagename{xskak}\footnote{\packagename{texmate} needs another
output.} and where I see problems. The description should also help
you to enter or convert complicated games manually.


\minisec{The PGN-infos}

The «tag pair» section of a chess game in PGN shouldn't pose problems
to a converter. The tags should be simply converted to the equivalent
keys of a \cs{newchessgame}:

\begin{lstlisting}
 [Round "29"]
 [White "Fischer, Robert J."]
 [Black "Spassky, Boris V."]
 [Result "1/2-1/2"]
\end{lstlisting}
%
\begin{lstlisting}[escapechar=/]
 \newchessgame[id=/\meta{GameId}/, %see below
               round={29},
               white={Fischer, Robert J.},
               black={Spassky, Boris V.},
               result={1/2-1/2}]
\end{lstlisting}

\xskakset{style=UF}%
If the game use a \texttt{SetUp} and a \texttt{FEN} tag to set an
alternative starting position, the converter will have to extract the
start move from the FEN and set it with the key \key{moveid}.

\minisec{The \meta{GameId}}

While it is probably possible to give each game an unique
\meta{GameId} I don't think that it is sensible or really necessary.
I would suggest, that each game gets uses\\
\verb+\newchessgame[id=main\xskakaddtoid,...]+ where
\cs{xskakaddtoid} is empty by default but can be redefined to allow
some automatic numbering.

\minisec{The parsing command}

As you will probably sometimes use \cs{mainline} and sometimes
\cs{hidemoves} it is probably best if the converter use an general
\cs{parsechessgame} which can be \cs{let} to both commands.

\minisec{NAG's}

Most NAG's don't need any special handling as \packagename{xskak} can
parse them.

But some chess applications seems to put NAG's \emph{after} a move to
denote a symbol that should go \emph{before} the move. E.g.
 \verb+1. e5 $142+
%
should become \betteris\variation{1. e5}.

Putting such symbols before a move number is in most cases only
possible at the start of a \cs{mainline} or \cs{variation} so the
converter will have probably to split the input and add the symbol
either directly or through a style. A converter will need a list of
all this special cases and good rules how to handle them.




\minisec{Long comments}

Longer commands (in the PGN in curly braces) could be handled  either
by adding a \cs{xskakcomment} at the begin of the comment or by
ending the parsing and starting it again after the comment. The first
method has the advantage that one can use styles to format to some
extend the comments (or gobble them) but can cause problems if the
comment contains something that isn't allowed in the argument of a
command.

I think some real examples are needed to decide what is the best way.


\minisec{The content of long comments}

I don't think that is possible to handle all possible contents
automatically and get a  suitable result for \LaTeX. You will either
have to follow some rules while inputing the comments or tidy them up
later.

But a converter will probably be able to identify moves like
\texttt{4.\,\,Nf3} and to surround them by suitable
\verb+\variation[style=...,level=...]{...}+.

\minisec{Diagrams}

Some chess applications let you put markers at places where you would
like to print a diagram. E.g. chessbase put
\texttt{Diagramm~\#}\footnote{I don't know if «Diagramm» is language
dependent. Probably yes.} in a comment.

A converter should also be able to recognize such markers for
diagrams. But it is not easy to decide what should be done with this
markers. In some cases you would probably not want to print boards at
all. Sometimes the game should be interrupted and the diagram printed
directly. Sometimes it should float. And I recall that someone once
tried to put all diagrams in the order there were mentioned in the
margin. If such markers are used it is probably the best if the
converter sets a tag and sets one of the fake NAG's \texttt{\$D} and
\texttt{\$d}. Then \cs{printchessgame} and the list of diagrams could
be used to print the moves and the diagrams in the intended order
(but it will probably need some coding to get the right results). The
tag and the NAG should be move outside of the comment as the comment
isn't processed in all cases.

\begin{lstlisting}[caption={Input and suggested output for diagram
markers}]
 10. c3 e5 $1 {Diagramm # Die energische und konsequente
 Fortsetzung, Schwarz öffnet das Zentrum, um seine Figuren zu
 aktivieren.} 11. h3
\end{lstlisting}

\begin{lstlisting}
 \mainline{10. c3 e5\xskakset{tag=dia1} $d $1
  \xskakcomment{ Die energische und konsequente Fortsetzung,
 Schwarz öffnet das Zentrum, um seine Figuren zu aktivieren. }
 11. h3}
\end{lstlisting}




\minisec{Variations}

Here starts the real problems. Even if we forget for the moment the
parsing of moves and diagrams it is not easy to describe a suitable
output. The main problems come from the fact that styles and good
typesetting are involved: E.g. PGN code two variations in a row as
\texttt{(1...~e5) (1...~e6)}. But you would probably print this as
(\variation{1... e5}; \variation{1... e6}).

So the converter will have to look ahead and use suitable «open»,
«middle», «close» and «single» styles  in the different combinations.

If you make long analyzes you will probably want to use numbered list
for  the variations  -- but not every variation is so important that
it need its own number. In this case there must be a marker that
tells the converter to handle this variation differently.

I haven't solved all this problems but I think that the following
code could be used as starting point for further discussion about a
\LaTeX-chess input\footnote{If one really wants to parse all
variations a better  naming for the nodes is needed.}:

\begin{lstlisting}
 \newchessgame[id=main, ...]
 \mainline{...}
  %new var level
  \newchessgame[newvar=main,id=var1a]%
  \mainline[level=1open]{.....}
  %same var level
  \newchessgame[newvar=main,id=var1b]%
  \mainline[level=1close]{....}
    %new var level
    \newchessgame[newvar=var1b,id=var2a]%
    \mainline[level=2single]{....}
  %% return to var1b:
  \resumechessgame[id=var1b]
  \mainline[level=1open]{....}
\end{lstlisting}


\minisec{Initialization and title}

Until now I have only discussed the game itself. But you would
probably want to print titles, make table of contents and other
lists. So one should also add some generic commands or environments
like \verb+\begin{chessgame}+/\hspace{0pt}\verb+\end{chessgame}+ and
\verb+{\chessgameheader}+ to each game.






\section{Compability issues}
\subsection{\texorpdfstring{\packagename{xskak} and
\packagename{texmate}}{xskak and texmate}}\index{texmate}

It will probably work fine to use both packages in one document.

\cs{newchessgame} and \cs{resumechessgame} will both internally issue
a \cs{fenposition} command that will setup the games also for
\packagename{texmate}. This naturally means in return that both
commands interfere with \packagename{texmate}!

You can't benefit of the extended input possibilities for NAG's and
comments. \packagename{texmate} has its own parsing method.

Btw: If you are using  \packagename{texmate} you should always indent
your input as the package doesn't recognize new lines as spaces.

\subsection{\texorpdfstring{\packagename{xskak} and
\packagename{beamer}}{xskak and beamer}}\index{beamer}

When making presentations with \packagename{beamer} you should
remember two things: First you can't use verbatim material or other
commands that relies on catcode changes in the
\texttt{frame}-environment or -command. That means for \skaksty that
you can't use \# in a \cs{mainline} but should replace it by
\cs{mate}.

Second, if you are using overlays the content of the frame is
processed more than once. That means that you should be careful to
reset counters if you don't want to run into never ending loops.

The following listing shows an example how to print a chessgame with
beamer. If your pdf-reader can handle annotations you can see the
result in the attached file: \attachfile{xskak_and_beamer.pdf} It
uses overlays (which worked fine) and also shows how you can handle
games in another language:

\lstinputlisting{xskak_and_beamer.tex}

\phantomsection
\makeatletter\def\index@prologue{\section*{Index}\addcontentsline{toc}{section}{Index}}\makeatother
\printindex
\end{document}

\end{document}
